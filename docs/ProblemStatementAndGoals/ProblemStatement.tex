\documentclass{article}

\usepackage{tabularx}
\usepackage{booktabs}
\usepackage{graphicx}
\usepackage{amssymb}
\usepackage{color}

\title{Problem Statement and Goals\\ Digital Twin Forest}

\author{Team 8\\Jiacheng Wu, Yichen Jiang, Tingyu Shi, Bowen Zhang, Junhong Chen}

\date{\today}

\begin{document}

\maketitle

\begin{table}[hp]
\centering
\caption{Revision History} \label{TblRevisionHistory}
\begin{tabularx}{11cm}{X X X}
\toprule
\textbf{Date} & \textbf{Developer(s)} & \textbf{Change}\\
\midrule
September 16,2022 & Jiacheng Wu, Yichen Jiang, Tingyu Shi, Bowen Zhang, Junhong Chen & First version of problem statement and goals\\
\hline
September 17,2022 & Jiacheng Wu, Yichen Jiang, Tingyu Shi, Bowen Zhang, Junhong Chen & Complete stretch goals\\
\bottomrule
\end{tabularx}
\end{table}

\section{Problem Statement}
\subsection{Problem}
A digital twin forest is a virtual representation of the natural world, specifically a real forest. By taking the distribution, ages, average height and leaf density of the forest, we are able to model the forest with the data we collected so that we can supervise the forest from a distance and collect data from fire, cutting, climate, etc. so that we can predict the impact those events have to the forest. This project can be beneficial for both commercial and scientific use. For commercial use, the product can help forest owners make decisions, and for scientific use, the product can help researchers to study climate change.

\subsection{Inputs and Outputs}

Inputs:
\begin{itemize}
    \item Collected data related to target forest from the lab. 
    \item Scanning data from the target natural trees.
\end{itemize}
Outputs:
\begin{itemize}
    \item Virtual representations of the target natural forest.
    \item Integration of forest data with virtual forest representations.
\end{itemize}


\subsection{Stakeholders}
\begin{itemize}
    \item Dr. Alemu Gonsamo from School of Earth, Environment and Society McMaster University (Dr. Gonsamo is the supervisor of this project.)
    \item Forest Owners(The final project can be helpful for forest to better manage the 
    forest and make decisions)
    \item Climate change researchers(The final product can be helpful for researchers to 
    study climate change)
\end{itemize}
\subsection{Environment}
\begin{itemize}
    \item iPad Pro LiDAR and multi-angle smart phone scans with photogrammetry technique: The camera that scans the trees and generate point cloud for future modeling.
    \item Laptops with Mac OS system: The devices used for modeling, coding and testing.
    \item 3D Scanner app: The app generates the 3D-reconstruction of the environment. It provides the team with the basic data of the forest, such as the height and diameters.
    \item Agisoft Metashape: A software that performs photogrammetric processing of digital images. It generates the models and data that the team needs to represent the virtual forest.
    \item Unity: A game engine that support augmented reality development and model editing. The team will build the virtual forest, and design the user interface here.
    \item Virtual Studio 2019: The IDE for augmented reality implementation. It supports C Sharp auto-correction and in game tests, so it has been widely used in the field of AR development.
    \item Xcode: The IDE that runs on the Mac OS system. The team will compile the final application with it and test the app with IOS devices.
\end{itemize}

\section{Goals}
\begin{itemize}
    \item Implement the virtual forest, which corresponds to the target natural forest. The model of a single tree is obtained by LiDAR scanning on the field. The final project combines previous models and lab statistics to give a virtual view of the forest. 
    \item Provide basic representation of data, such as age, height, and density of the trees. 
\end{itemize}

\section{Stretch Goals}

\begin{itemize}
    \item Represent overall data of forest, like amount of logging, the situation of growth.
    \item Record significant data for later use, supporting forest owners to optimize their strategy to manage the forest.
    \item Support the climate and forest researchers to predict certain values related to the forest, including the situation of 
    soil, percentage of carbon dioxide, etc. to better study climate change in this forest.

\end{itemize}






\end{document}
