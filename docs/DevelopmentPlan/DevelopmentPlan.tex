\documentclass{article}

\usepackage{booktabs}
\usepackage{tabularx}
\usepackage{ulem}
\usepackage[colorlinks,linkcolor=blue]{hyperref}

\title{Development Plan\\ Digital Twin Forest}

\author{Team 8\\Jiacheng Wu, Yichen Jiang, Tingyu Shi, Bowen Zhang, Junhong Chen}

\date{\today}

\begin{document}
 
\maketitle

\begin{table}[hp]
\centering
\caption{Revision History} \label{TblRevisionHistory}
\begin{tabularx}{\textwidth}{XXX}
\toprule
\textbf{Date} & \textbf{Developer(s)} & \textbf{Change}\\
\midrule
Sept 17, 2022 & All team members & Revision 0\\
\hline
Sept 26, 2022 & All team members & Modify technology\\
\hline
March 29, 2023 & All team members & Final Version\\
\bottomrule
\end{tabularx}
\end{table}
\section{Team Meeting Plan}
\begin{itemize}
    \item Time\\ Every Thursday from 1:30 PM to 3:30 PM.
    \item Location \\ If the team needs to work with Dr. Gonsamo's lab members, the meeting room
    will be the lab of room 316 of the General Science Building. Otherwise, the meeting will be hosted in H.G. Thode Library.
    \item Frequency \\ Weekly. (Multiple meetings will be held within one week if necessary.)
    \item Contents of meeting\\
    The contents of meetings include discussion, implementation, assigning work outside 
    the meetings and reporting working progress. All the decisions and modifications
    should be recorded and updated by the end of the day of meetings.
    \item Roles\\ 
    Bowen Zhang: Host all the meetings and gather participants' ideas.\\
    Tingyu Shi: Record and document meeting contents.\\
    Jiacheng Wu: Participant.\\
    Yichen Jiang: Participant.\\
    Junhong Chen: Participant.
    \item Rules for agendas\\
    An agenda is supposed to be made before every meeting. The following are rules for
    meeting agendas:
    \begin{itemize}
    \item All team members are supposed to attend the meetings. If any member cannot attend the meeting, he/she needs to tell the other teammates at least one day before the meeting.
    \item Topics must be determined before every meeting.
    \item The duration of the meeting should be established prior to its commencement.
    \item There is only one chair/leader in the meeting.
    \item Conflicts in the meeting shall be recorded and solved before the next meeting.
    \item All members should communicate respectfully during the meeting.
    \item All members shall be assigned a “take home” work after the meeting. The estimated time of the work of each member shall be close.
    \item First topic will be reviewing the agenda.
    \item Assess all team members’ contributions in every meeting.
    \end{itemize}
\end{itemize}

\section{Team Communication Plan}
Communication within our group will be completed through weekly in-person meetings, Microsoft Teams, emails, and informal daily communications. 

\section{Team Member Roles}
\begin{itemize}
    \item Jiacheng Wu: Unity Developer
    \item Yichen Jiang: UI designer
    \item Tingyu Shi: Unity Developer, Latex Expert
    \item Bowen Zhang: Forest modeling, Liaison
    \item Junhong Chen: UI designer, coder, PhotoShop expert
\end{itemize}

\newpage

\section{Workflow Plan}
\begin{itemize}

\item We will use Unity for the majority of the development process. We will use GitHub to deliver, store, and
synchronize all the Unity scripts. The latest version of the Unity project will be updated during the weekly meetings. Then the team will push the latest scripts to GitHub with corresponding tags, which include folders of assets, user interface, and user settings. We would also utilize issues posted by other teams on GitHub to record and track any suggestions on our project and document.
 
	\item For the documentation required in the development process, we will follow the template provided on the course's GitLab page. For any documentation, we will use overleaf for \LaTeX writing to accomplish real-time communication and synchronized collaboration. After the documents are finished, we push documents to our GitHub with corresponding comments such as Rev0 and Rev1.
\end{itemize}

\section{Proof of Concept Demonstration Plan}
\subsection{Risks}
\begin{itemize}
\item Creating huge amounts of trees efficiently.
\item How to ensure the fidelity of the model
\item How to simulate the first perspective of a real 
person to view the forest.
\item Different assets used in Unity may require different Unity versions or rendering pipelines.
\item As the plant density increases, it uses more and more draw calls which is not affordable.
\item How to design the UI to visualize all the forest 
data efficiently.
\end{itemize}


\subsection{POC Demonstration contents}
\begin{itemize}
\item The POC demonstration should include a virtual
representation of a limited amount of trees.
\item POC should have a working UI to present all the 
forest. 
\item POC should demonstrate a way to view the forest 
from a first-person view.
\end{itemize}

\newpage

\section{Technology}
\begin{itemize}
\item programming language: The whole project will be
implemented in C Sharp language.
\item Linter tool: The team will use Visual Studio 2019 as
the main IDE.
\item Unit Testing Tool: the team will not do automatic tests because all the tests will be done manually in unity, and require the whole system to run.
\item CI plans: We will not use continuous integration in
our project because the majority of our project will be
finished in Unity. For each iteration or any change, we
would choose to test the packages in Unity instead of
using automatic testing. We need to make sure the latest
version of the model can be loaded and give a satisfying
view of the forest, so the automatic testing may not
satisfy our requirements in this project.
\item Performance measuring tools: Unity will be used to
record the number of frames per second of the final
project. 

\item The following are possible libraries:
\begin{itemize}
\item UnityEngine.UI
\item UnityEngine
\item Terrain Tool
\item Tree Editor
\item Scene Manager
\item Newtonsoft
\item Aura
\end{itemize}

\end{itemize}

\section{Coding Standard}
\subsection{Program File Names}
The file names will be written in Pascal style, and the titles will be descriptive.

\subsection{C Sharp}
The coding style of the C sharp language that the team will adopt is the \href{https://learn.microsoft.com/en-us/dotnet/csharp/fundamentals/coding-style/coding-conventions}{Microsoft C\# Coding Conventions}. The variable names will be short but descriptive. All variables will be written in camel case. For example, the height of a tree should be written as $treeHeight$.

\subsection{Unity File Names}
Unity files such as images, videos, prefabs, and scene files will be written in lowercase. The team will use $-$ to replace the spaces in file names for readability.   

\section{Project Scheduling}
\begin{itemize}
    \item Technology: Shared MS Excel Gantt Chart
    \item Major Milestones: 
    \begin{itemize}
        \item SRS Rev0 (October 5)
        \item Hazard Analysis 0 (October 19)
        \item V\&V Plan Rev0 (November 2)
        \item POC Demo (November 14-25)
        \item Design Document Rev0 (January 18)
        \item Demonstration Rev0 (February 6- 17)
        \item V\&V Report Rev0  (March 8)
        \item Final Demo (March 20 - 31)
        \item Final Documentation (April 5)
    \end{itemize}
    \item Task Decomposition Plan:\\ We will discuss this plan with Dr. Gonsamo and his lab
    members first. Then the team leader will allocate tasks according to team members' roles. 
    Our current plan is to decompose this project into three parts: measuring data, modeling, and coding.
    \item Please check our project schedule \href{https://github.com/wuj187/DigitalTwinCAS/tree/main/docs/DevelopmentPlan/Project_Schedule}{here}.
\end{itemize}
\end{document}