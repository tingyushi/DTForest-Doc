\documentclass{article}

\usepackage{booktabs}
\usepackage{tabularx}
\usepackage{ulem}
\usepackage[colorlinks,linkcolor=blue]{hyperref}

\title{Development Plan\\ Digital Twin Forest}

\author{Team 8\\Jiacheng Wu, Yichen Jiang, Tingyu Shi, Bowen Zhang, Junhong Chen}

\date{\today}

%%% Comments

\usepackage{color}

\newif\ifcomments\commentstrue %displays comments
%\newif\ifcomments\commentsfalse %so that comments do not display

\ifcomments
\newcommand{\authornote}[3]{\textcolor{#1}{[#3 ---#2]}}
\newcommand{\todo}[1]{\textcolor{red}{[TODO: #1]}}
\else
\newcommand{\authornote}[3]{}
\newcommand{\todo}[1]{}
\fi

\newcommand{\wss}[1]{\authornote{blue}{SS}{#1}} 
\newcommand{\plt}[1]{\authornote{magenta}{TPLT}{#1}} %For explanation of the template
\newcommand{\an}[1]{\authornote{cyan}{Author}{#1}}
%%% Common Parts

\newcommand{\progname}{ProgName} % PUT YOUR PROGRAM NAME HERE
\newcommand{\authname}{Team \#, Team Name
\\ Student 1 name and macid
\\ Student 2 name and macid
\\ Student 3 name and macid
\\ Student 4 name and macid} % AUTHOR NAMES                  

\usepackage{hyperref}
    \hypersetup{colorlinks=true, linkcolor=blue, citecolor=blue, filecolor=blue,
                urlcolor=blue, unicode=false}
    \urlstyle{same}
                                


\begin{document}
 
\maketitle

\begin{table}[hp]
\centering
\caption{Revision History} \label{TblRevisionHistory}
\begin{tabularx}{\textwidth}{XXX}
\toprule
\textbf{Date} & \textbf{Developer(s)} & \textbf{Change}\\
\midrule
Sept 17, 2022 & All team members & Revision 0\\
Sept 26, 2022 & All team members & Modify technology\\
\bottomrule
\end{tabularx}
\end{table}
\section{Team Meeting Plan}
\begin{itemize}
    \item Time\\ Every Thursday from 1:30 PM to 3:30 PM.
    \item Location \\ If the team needs to work with Dr. Gonsamo's lab members, the meeting room
    will be the lab of room 316 of General Science Building. Otherwise, the meeting will be hosted in H.G. Thode Library.
    \item Frequency \\ Weekly. (Multiple meetings will be held within one week if necessary.)
    \item Contents of meeting\\
    The contents of meetings include discussion, implementation, assigning works outside 
    the meetings and reporting working progress. All the decisions and modifications
    should be recorded and updated by the end of the day of meetings.
    \item Roles\\ 
    Bowen Zhang: Host all the meetings and gather participants' ideas.\\
    Tingyu Shi: Record and document meeting contents.\\
    Jiacheng Wu: Participant.\\
    Yichen Jiang: Participant.\\
    Junhong Chen: Participant.
    \item Rules for agendas\\
    An agenda is supposed to be made before every meeting. The following are rules for
    meeting agendas:
    \begin{itemize}
    \item All team members are supposed to attend the meetings. If any member cannot attend the meeting, he/she needs to tell the other teammates at least one day before the meeting.
    \item Topics must be determined before every meeting.
    \item The length of meeting time shall be determined before the meeting.
    \item There is only one chair/leader in the meeting.
    \item Conflicts in the meeting shall be recorded and solved before the next meeting.
    \item All members should communicate respectfully during the meeting.
    \item All members shall be assigned a “take home” work after the meeting. The estimated time of the work of each member shall be close.
    \item First topic will be reviewing the agenda.
    \item Assess all team members’ contributions in every meetings.
    \end{itemize}
\end{itemize}

\section{Team Communication Plan}
Communication within our group will be completed through weekly in-person meetings, Microsoft Teams, emails, and informal daily communications. 

\section{Team Member Roles}
\begin{itemize}
    \item Jiacheng Wu: Developer, UI designer, C Sharp expert
    \item Yichen Jiang: Developer, Lead Tester
    \item Tingyu Shi: Developer, Latex Expert, Unity Expert
    \item Bowen Zhang: Developer, Liaison, Group Leader
    \item Junhong Chen: Developer, Git Expert
\end{itemize}

\section{Workflow Plan}

\begin{itemize}
	\item We will use Unity for majority of the modelling part. And we will use git to deliver, store, and synchronize the Unity packages. Specific division of workload should be discussed in the weekly meetings. And then each member pushes his or her own share of work into the git, which should include folders of assets, packages, project settings, user settings, and related license. We will use co author methods to push.
	\item For the documentation required in the development, we will follow the template provided in GitHub of the course. For any documentation, we will use overleaf and hold online meetings to realize real-time communication and synchronized collaboration. After the documents are finished, we push documents into git and sign every team member as a co-author. 
\end{itemize}

\section{Proof of Concept Demonstration Plan}
\subsection{Risks:}
\begin{itemize}
    %\item Mobile device may not provide sufficient computing capacities to test the models.\\\\ 
        %Solution: We might convert high polygon models to low polygon models when representing in the mobile device. 
    \item The accuracy of measurement and modelling strongly depends on the reality with different
    weather.\\Solution: Choose the timing with better condition.
        
    \item The ideal process of scanning and modelling would be scan an entire plot at once, while this
    process is limited by the scanning device we use, which would be cell phone and iPad.\\Solution:
    Scan single trees and splice the models in Unity.
    
    \item The trees might be too high to scan all aspects of the trees or result in a poor
    precision.\\Solution: Employ drones to scan the pictures.
\end{itemize}

\textcolor{red}{The following are revised Risks for POC Demo.}
\textcolor{red}{
\begin{itemize}
\item Creating huge amount of trees efficiently.
\item The virtual representation should be close to the real forest.
\item How to simulate the first perspective of a real person to view the forest.
\item Different assets used in Unity may require different Unity versions or rendering pipelines.
\item As the plant density increases, it uses more and more draw calls which is not affordable.
\end{itemize}
}    

\subsection{POC Demonstration contents}
\begin{itemize}
    \item \textcolor{red}{The POC demonstration should include a virtual representation of a limited
    amount of trees, along with a simple version of user interface. The data displayed shall or shall not
    be close to the real forest data, this depends on the data collection process by Dr.Gonsamo's lab. }

    \item \textcolor{red}{\sout{POC should include virtual representation of a limited amount
    of trees with qualified devices like laptops scanned by mobile devices with LiDAR like iPhone or iPad.
    This can prove that modelling and a virtual representation of forest could be realized.}}
    
    \item \textcolor{red}{\sout{POC should include trees higher than 5 meters to provide a proof of adequate measurement and modelling.}} 
    
    \item \textcolor{red}{POC should include the demonstration of our project prototype that can operate
    properly.}
    
    \item \textcolor{red}{POC should demonstrate a way to simulate first perspective of real persons to
    view the forest.}
\end{itemize}

\section{Technology}

\begin{itemize}
\item programming language: The whole project will be implemented in C Sharp language.
\item linter tool: The team will use Visual Studio 2019 as the main IDE.
\item unit testing framework: VS2019 will be used for the unit test. The team will create a unit test project (.NET Framework) that contains MSTest unit tests.
\item code coverage measuring tools: The JetBrains dotCover is a code coverage tool that integrates with VS2019. It can execute and run coverage analysis for unit tests in Visual Studio.
\item CI plans: We will not use continuous integration in our project, because majority of our project will be finished in Unity. For each iteration or any change, we would choose to test the packages in Unity instead of using the automatic testing. We need to make sure the latest version of model can be loaded and give a satisfying view of forest, so the automatic testing may not satisfy our requirements in this project.
\item performance measuring tools: Unity will be used to record the number of frames per second of the final project. 
%\item Libraries: AR SDKs like ARFoundation and Vuforia will be used for the development of the application. The packages can be imported through Unity and get called in VS2019.
\item Libraries: We will import several Unity packages for modelling and post processing. Possible plug-ins may include unity terrain tools package and universal render pipeline.
\item Tools: iPad Pro LiDAR and multi angle smart phone scans with photogrammetry technique will be used for 3D scanning. Agisoft metashape and 3D Scanning App will be used to generate 3D models and photogrammetric data.
\end{itemize}

\section{Coding Standard}
\subsection{Program File Names}
The file names will be written in Pascal style and the titles will be descriptive.

\subsection{C Sharp}
The coding style of the C sharp language that the team will adopt is the \href{https://learn.microsoft.com/en-us/dotnet/csharp/fundamentals/coding-style/coding-conventions}{Microsoft C\# Coding Conventions}. The variable names will be short but descriptive. All variables will be written in camel case. For example, the height of a tree should be written as $treeHeight$.

\subsection{Unity File Names}
Unity files such as images, videos, prefabs and scene files will be written in lowercase. The team will use $-$ to replace the spaces in file names for readability.   

\section{Project Scheduling}
\begin{itemize}
    \item Technology: Shared MS Excel Gantt Chart
    \item Major Milestones: 
    \begin{itemize}
        \item SRS Rev0 (October 5)
        \item Hazard Analysis 0 (October 19)
        \item V\&V Plan Rev0 (November 2)
        \item POC Demo (November 14-25)
        \item Design Document Rev0 (January 18)
        \item Demonstration Rev0 (February 6- 17)
        \item V\&V Report Rev0  (March 8)
        \item Final Demo (March 20 - 31)
        \item Final Documentation (April 5)
    \end{itemize}
    \item Task Decomposition Plan:\\ This plan will be discussed with Dr. Gonsamo and his lab
    members first. Then the team leader will allocate tasks according to team member roles. 
    Our current plan is decomposing this project into three parts, which are measuring data, modeling
    and coding.
    \item Please check our project schedule \href{https://github.com/wuj187/DigitalTwinCAS/tree/main/docs/DevelopmentPlan/Project_Schedule}{here}.
\end{itemize}
\end{document}