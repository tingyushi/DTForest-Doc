

\documentclass[12pt, titlepage]{article}

\usepackage{amsmath, mathtools}

\usepackage[round]{natbib}
\usepackage{amsfonts}
\usepackage{amssymb}
\usepackage{graphicx}
\usepackage{colortbl}
\usepackage{xr}
\usepackage{hyperref}
\usepackage{longtable}
\usepackage{xfrac}
\usepackage{tabularx}
\usepackage{float}
\usepackage{siunitx}
\usepackage{booktabs}
\usepackage{multirow}
\usepackage[section]{placeins}
\usepackage{caption}
\usepackage{fullpage}

\hypersetup{
bookmarks=true,     % show bookmarks bar?
colorlinks=true,       % false: boxed links; true: colored links
linkcolor=red,          % color of internal links (change box color with linkbordercolor)
citecolor=blue,      % color of links to bibliography
filecolor=magenta,  % color of file links
urlcolor=cyan          % color of external links
}

\usepackage{array}

%% Comments

\usepackage{color}

\newif\ifcomments\commentstrue %displays comments
%\newif\ifcomments\commentsfalse %so that comments do not display

\ifcomments
\newcommand{\authornote}[3]{\textcolor{#1}{[#3 ---#2]}}
\newcommand{\todo}[1]{\textcolor{red}{[TODO: #1]}}
\else
\newcommand{\authornote}[3]{}
\newcommand{\todo}[1]{}
\fi

\newcommand{\wss}[1]{\authornote{blue}{SS}{#1}} 
\newcommand{\plt}[1]{\authornote{magenta}{TPLT}{#1}} %For explanation of the template
\newcommand{\an}[1]{\authornote{cyan}{Author}{#1}}
%% Common Parts

\newcommand{\progname}{ProgName} % PUT YOUR PROGRAM NAME HERE
\newcommand{\authname}{Team \#, Team Name
\\ Student 1 name and macid
\\ Student 2 name and macid
\\ Student 3 name and macid
\\ Student 4 name and macid} % AUTHOR NAMES                  

\usepackage{hyperref}
    \hypersetup{colorlinks=true, linkcolor=blue, citecolor=blue, filecolor=blue,
                urlcolor=blue, unicode=false}
    \urlstyle{same}
                                


%%%%%%%%%% new command %%%%%%%%%%%%%%
\newcounter{acnum}
\newcommand{\actheacnum}{AC\theacnum}
\newcommand{\acref}[1]{AC\ref{#1}}

\newcounter{ucnum}
\newcommand{\uctheucnum}{UC\theucnum}
\newcommand{\uref}[1]{UC\ref{#1}}

\newcounter{mnum}
\newcommand{\mthemnum}{M\themnum}
\newcommand{\mref}[1]{M\ref{#1}}
%%%%%%%%%% new command end %%%%%%%%%%%%

\begin{document}

\title{Module Interface Specification for \progname{}}

\author{\authname}

\date{\today}

\maketitle

\pagenumbering{roman}

\section{Revision History}

\begin{tabularx}{\textwidth}{p{3cm}p{2cm}X}
\toprule {\bf Date} & {\bf Version} & {\bf Notes}\\
\midrule
Jan 14 & 1.0 & First Version\\
\hline
April 4 & 2.0 & Final Version\\
\bottomrule
\end{tabularx}

~\newpage

\section{Symbols, Abbreviations and Acronyms}

See SRS Documentation at \href{https://github.com/wuj187/DigitalTwinCAS/blob/main/docs/DocRevision/SRSRevision/SRSRevision.pdf}{here}.
\\

\begin{tabular}{l l} 
  \toprule		
  \textbf{symbol} & \textbf{description}\\
  \midrule 
  AC & Anticipated Change\\
  DAG & Directed Acyclic Graph \\
  DBH & Diameter at breast height\\
  FR & Functional Requirement\\
  GUI & Graphical User Interface\\
  LAI & Leaf Area Index\\
  M & Module \\
  MG & Module Guide \\
  MVC & Model, Viewer, Controller\\
  NFR & Non-Functional Requirement\\
  OS & Operating System \\
  R & Requirement\\
  SC & Scientific Computing \\
  SRS & Software Requirements Specification\\
  UC & Unlikely Change \\
  \bottomrule
\end{tabular}\\
\newpage

\tableofcontents

\newpage

\pagenumbering{arabic}

\section{Introduction}

The following document outlines the Module Interface 
Specifications for \progname{}, which is a virtual 
representation of the real world, including physical objects, 
processes, relationships, and behaviors. Components of a 
digital twin encompass data capture
and integration, visualization, and advanced analysis 
including AI, automation, information sharing and 
collaboration. This project can benefit two groups 
of users: forest owners who can utilize the software for forest management and meteorologists who can use it for research purposes. 

\noindent Complementary documents include the \href{https://github.com/wuj187/DigitalTwinCAS/blob/main/docs/SRS/SRS.pdf}{System 
Requirement Specifications} and \href{https://github.com/wuj187/DigitalTwinCAS/blob/main/docs/Design/Software-Architecture-Design/MG.pdf}{Module Guide}. 

\section{Notation}

\noindent The structure of the MIS for modules comes from \citet{HoffmanAndStrooper1995},
with the addition that template modules have been adapted from
\cite{GhezziEtAl2003}.  The mathematical notation comes from 
Chapter 3 of
\citet{HoffmanAndStrooper1995}.  For instance, the symbol := 
is used for a multiple assignment statement and conditional 
rules follow the form $(c_1 \Rightarrow r_1 | c_2 \Rightarrow
r_2 | ... | c_n \Rightarrow r_n )$.

The following table summarizes the primitive data types used by \progname. 

\begin{center}
\renewcommand{\arraystretch}{1.2}
\noindent 
\begin{tabular}{l l p{7.5cm}} 
\toprule 
\textbf{Data Type} & \textbf{Notation} & \textbf{Description}\\ 
\midrule
string & String & a sequence of characters\\
character & char & a single symbol or digit\\
integer & $\mathbb{Z}$ & a number without a fractional component in (-$\infty$, $\infty$) \\
natural number & $\mathbb{N}$ & a number without a fractional component in [1, $\infty$) \\
real & $\mathbb{R}$ & any number in (-$\infty$, $\infty$)\\
Boolean & Boolean & a value that takes either True or False\\
\bottomrule
\end{tabular} 
\end{center}

\noindent
The specification of \progname \ uses some derived data types: sequences, strings, and
tuples. Sequences are lists filled with elements of the same data type. Strings
are sequences of characters. Tuples contain a list of values, potentially of
different types. In addition, \progname \ uses functions, which
are defined by the data types of their inputs and outputs. Local functions are
described by giving their type signature followed by their specification.

\newpage

\section{Module Decomposition}


\begin{table}[H]
\caption{Module Hierarchy(First Controller Table)}
\label{TblControllers}

\centering
\begin{tabular}{p{0.3\textwidth} p{0.6\textwidth}}
\toprule
\textbf{Level 1} & \textbf{Level 2}\\
\midrule

\multirow{18}{0.3\textwidth}{Controller Modules}
& \refstepcounter{mnum} \mthemnum \label{Controller1}: JsonFileReader \\
& \refstepcounter{mnum} \mthemnum \label{Controller2}: JsonFileWriter  \\
& \refstepcounter{mnum} \mthemnum \label{Controller3}: PauseManager  \\
& \refstepcounter{mnum} \mthemnum \label{Controller4}: PlayerMovement  \\
& \refstepcounter{mnum} \mthemnum \label{Controller5}: NewDataInputBoxController\\
& \refstepcounter{mnum} \mthemnum \label{Controller6}: StartButtonController  \\
& \refstepcounter{mnum} \mthemnum \label{Controller7}: InstructionButtonController \\
& \refstepcounter{mnum} \mthemnum \label{Controller8}: UpdateDataButtonController\\
& \refstepcounter{mnum} \mthemnum \label{Controller9}: ContactUsButtonController \\
& \refstepcounter{mnum} \mthemnum \label{Controller10}: QuitButtonController  \\
& \refstepcounter{mnum} \mthemnum \label{Controller11}: BackButtonController \\
& \refstepcounter{mnum} \mthemnum \label{Controller12}: PlotSelectionDropDownController \\
& \refstepcounter{mnum} \mthemnum \label{Controller13}: TreeTypeSelectionDropDownController \\
& \refstepcounter{mnum} \mthemnum \label{Controller14}: ShowEnvDataButtonController \\
& \refstepcounter{mnum} \mthemnum \label{Controller15}: ShowTreeParamButtonController \\
& \refstepcounter{mnum} \mthemnum \label{Controller16}: EnvDataSelectionButtonController\\
& \refstepcounter{mnum} \mthemnum \label{Controller17}: DataTypeSelectionButtonsController \\
& \refstepcounter{mnum} \mthemnum \label{Controller18}: SaveButtonController \\
\bottomrule

\end{tabular}

\end{table}

\newpage
\begin{table}[H]
\caption{Module Hierarchy(Second Controller Table)}
\label{TblControllers}

\centering
\begin{tabular}{p{0.3\textwidth} p{0.6\textwidth}}
\toprule
\textbf{Level 1} & \textbf{Level 2}\\
\midrule

\multirow{18}{0.3\textwidth}{Controller Modules}

& \refstepcounter{mnum} \mthemnum \label{Controller19}: FileWriter \\
& \refstepcounter{mnum} \mthemnum \label{Controller20}: FileReader \\
& \refstepcounter{mnum} \mthemnum \label{Controller21}: SeasonChangeController \\
& \refstepcounter{mnum} \mthemnum \label{Controller22}: movePanelController \\
& \refstepcounter{mnum} \mthemnum \label{Controller23}: treePlantingController \\
& \refstepcounter{mnum} \mthemnum \label{Controller24}: TreeSwitchButtonController \\
& \refstepcounter{mnum} \mthemnum \label{Controller25}: pieChartButtonController \\
& \refstepcounter{mnum} \mthemnum \label{Controller26}: pieChartController \\
& \refstepcounter{mnum} \mthemnum \label{Controller27}: SeasonChangeButtonController \\
& \refstepcounter{mnum} \mthemnum \label{Controller28}: TreeSwitchController \\
\bottomrule

\end{tabular}

\end{table}

\newpage

\begin{table}[H]
\caption{Module Hierarchy(Models)}
\label{TblModels}

\centering
\begin{tabular}{p{0.3\textwidth} p{0.6\textwidth}}
\toprule
\textbf{Level 1} & \textbf{Level 2}\\
\midrule

\multirow{14}{0.3\textwidth}{Model Modules}
& \refstepcounter{mnum} \mthemnum \label{Model1}: ForestTrees \\
& \refstepcounter{mnum} \mthemnum \label{Model2}: ForestSky  \\
& \refstepcounter{mnum} \mthemnum \label{Model3}: ForestTerrain \\
& \refstepcounter{mnum} \mthemnum \label{Model4}: RedPine  \\
& \refstepcounter{mnum} \mthemnum \label{Model5}: Oak   \\
& \refstepcounter{mnum} \mthemnum \label{Model6}: Beech  \\
& \refstepcounter{mnum} \mthemnum \label{Model7}: Birch  \\ 
& \refstepcounter{mnum} \mthemnum \label{Model8}: WhitePine  \\
& \refstepcounter{mnum} \mthemnum \label{Model9}: RedMaple \\
& \refstepcounter{mnum} \mthemnum \label{Model10}: RedOak  \\
& \refstepcounter{mnum} \mthemnum \label{Model11}: EnvData  \\
& \refstepcounter{mnum} \mthemnum \label{Model12}: PlotData  \\
& \refstepcounter{mnum} \mthemnum \label{Model13}: FirstPersonPlayer  \\
& \refstepcounter{mnum} \mthemnum \label{Model14}: JsonFile  \\
\bottomrule

\end{tabular}

\end{table}


\newpage

\begin{table}[H]
\caption{Module Hierarchy(First Viewers Table)}
\label{TblViewers1}

\centering
\begin{tabular}{p{0.3\textwidth} p{0.6\textwidth}}
\toprule
\textbf{Level 1} & \textbf{Level 2}\\
\midrule

\multirow{13}{0.3\textwidth}{Viewer Modules}
& \refstepcounter{mnum} \mthemnum \label{Viwer1}: MainPageDisplay \\
& \refstepcounter{mnum} \mthemnum \label{Viwer2}: StartButton \\
& \refstepcounter{mnum} \mthemnum \label{Viwer3}: InstructionButton \\
& \refstepcounter{mnum} \mthemnum \label{Viwer4}: ContactUsButton \\
& \refstepcounter{mnum} \mthemnum \label{Viwer5}: QuitButton \\
& \refstepcounter{mnum} \mthemnum \label{Viwer6}: InstructionInfoDisplay \\
& \refstepcounter{mnum} \mthemnum \label{Viwer7}: ContactUsInfoDisplay \\ 
& \refstepcounter{mnum} \mthemnum \label{Viwer8}: BackButton \\
& \refstepcounter{mnum} \mthemnum \label{Viwer9}: UpdateDataDisplay \\
& \refstepcounter{mnum} \mthemnum \label{Viwer10}: EnvDataSelectionButton \\
& \refstepcounter{mnum} \mthemnum \label{Viwer11}: DataTypeSelectionButtons \\
& \refstepcounter{mnum} \mthemnum \label{Viwer12}: NewDataInputBox \\
& \refstepcounter{mnum} \mthemnum \label{Viwer13}: SaveButton \\

\bottomrule
\end{tabular}
\end{table}

\newpage

\begin{table}[H]
\caption{Module Hierarchy(Second Viewers Table)}
\label{TblViewers2}

\centering
\begin{tabular}{p{0.3\textwidth} p{0.6\textwidth}}
\toprule
\textbf{Level 1} & \textbf{Level 2}\\
\midrule

\multirow{10}{0.3\textwidth}{Viewer Modules}
& \refstepcounter{mnum} \mthemnum \label{Viwer14}: CurrentDataDisplay \\
& \refstepcounter{mnum} \mthemnum \label{Viwer15}: PlotSelectionDropDown \\
& \refstepcounter{mnum} \mthemnum \label{Viwer16}: TreeTypeSelectionDropDown \\
& \refstepcounter{mnum} \mthemnum \label{Viwer17}: UpdateDataButton \\
& \refstepcounter{mnum} \mthemnum \label{Viwer18}: ForestDisplay \\
& \refstepcounter{mnum} \mthemnum \label{Viwer19}: ShowEnvDataButton \\
& \refstepcounter{mnum} \mthemnum \label{Viwer20}: ShowTreeParamButton \\
& \refstepcounter{mnum} \mthemnum \label{Viwer21}: EnvDataDisplay \\
& \refstepcounter{mnum} \mthemnum \label{Viwer22}: TreeParamDisplay \\
& \refstepcounter{mnum} \mthemnum \label{Viwer23}: PauseIndicatorDisplay \\
& \refstepcounter{mnum} \mthemnum \label{Viwer24}: SeasonChangeButton \\
& \refstepcounter{mnum} \mthemnum \label{Viwer25}: pieChartButton \\
& \refstepcounter{mnum} \mthemnum \label{Viwer26}: TreeSwitchButton \\
\bottomrule
\end{tabular}
\end{table}

\newpage
\newcommand{\bref}{\href{https://docs.unity3d.com/Packages/com.unity.ugui@1.0/manual/script-Button.html}{here}}
\section{MIS of Json File Reader (\mref{Controller1})} 

\subsection{Module}
JsonFileReader
\subsection{Uses}
UnityEngine\\
System.IO\\
UnityEngine.UI\\
\mref{Viwer22}\\
\mref{Viwer21}\\
\mref{Controller23}\\
\mref{Controller26}\\
\mref{Model12}
\subsection{Syntax}

\subsubsection{Exported Constants}
None

\subsubsection{Exported Access Programs}
\begin{center}
\begin{tabular}{|p{3cm}|p{4cm}|p{4cm}| p{5cm}|}
\hline
\textbf{Name} & \textbf{In} & \textbf{Out} & \textbf{Exceptions} \\
\hline
 Awake& &  & \\
\hline
 Start& &  & \\
\hline
 readfile& $\mathbb{Z}$ &  & \\
\hline
\end{tabular}
\end{center}

\subsection{Semantics}

\subsubsection{State Variables}
 treeParamDisplay: TreeParamDisplay\\
 envDataDisplay: EnvDataDisplay\\
 graphMaker: pieChartMaker\\
treePlanter: treePlanting\\
DataModelObj: DataModel\\
JsonModelObj: JsonModel\\
plotNumber: $\mathbb{Z}$\\
filePath: string\\
plotJsonData: string\\
\subsubsection{Environment Variables}
overalldata.json\\
plot1data.json\\
plot2data.json\\
plot3data.json\\
plot4data.json\\
plot5data.json\\
plot6data.json\\
plot7data.json\\
plot8data.json\\
plot9data.json\\
plot10data.json\\
plot11data.json\\
plot12data.json\\
plot13data.json\\
plot14data.json

\subsubsection{Assumptions}
None

\subsubsection{Access Routine Semantics}
Awake():
\begin{itemize}
    \item transition: readfile(0)
    \item output: None
    \item exception: None
\end{itemize}
Start():
\begin{itemize}
    \item transition: None
    \item output: None
    \item exception: None
\end{itemize}
readfile(value):
\begin{itemize}
    \item transition: Open the JSON file according to value, read all the content from the JSON file, update tree parameters of treeParamDisplay, update environemntal data in envDataDisplay, invoke the markChart() function of graphMaker, and invoke plantTrees() function of treePlanter 
    \item output: None
    \item exception: None
\end{itemize}
\subsubsection{Local Functions}
None
\newpage

\section{MIS of Json File Writer (\mref{Controller2})} 

\subsection{Module}
JsonFileWriter
\subsection{Uses}
UnityEngine\\
TMPro\\
Text\\
UnityEngine.UI\\
\mref{Controller1}
\subsection{Syntax}

\subsubsection{Exported Constants}
None

\subsubsection{Exported Access Programs}
\begin{center}
\begin{tabular}{|p{3cm}|p{4cm}|p{4cm}| p{5cm}|}
\hline
\textbf{Name} & \textbf{In} & \textbf{Out} & \textbf{Exceptions} \\
\hline
 Awake& &  & \\
\hline
 updateHandler& &  & InvalidInputException\\
\hline
 writeFile& String&  & \\
\hline
 removeMsg& &  & \\
\hline
 isValid& String&  Boolean& \\
\hline
 isIN&char, List\textless char\textgreater &Boolean  & \\
\hline
\end{tabular}
\end{center}

\subsection{Semantics}

\subsubsection{State Variables}
 indicator: ValueIndicator\\
 inputField: TMP\_InputField\\
notify: Text \\
FR: FileReader 
\subsubsection{Environment Variables}
overalldata.json\\
plot1data.json\\
plot2data.json\\
plot3data.json\\
plot4data.json\\
plot5data.json\\
plot6data.json\\
plot7data.json\\
plot8data.json\\
plot9data.json\\
plot10data.json\\
plot11data.json\\
plot12data.json\\
plot13data.json\\
plot14data.json

\subsubsection{Assumptions}
None

\subsubsection{Access Routine Semantics}
Awake():
\begin{itemize}
    \item transition: Invoke removeMsg()
    \item output: None
    \item exception: None
\end{itemize}
updateHandler():
\begin{itemize}
    \item transition: Validate the input of the text field and invoke writeFile function
    \item output: None
    \item exception: Throw InvalidInputException if the inputs are invalid
\end{itemize}
writeFile(newValue):
\begin{itemize}
    \item transition: Import the data from the text field and use it to rewrite the data stored in the JSON files based on the attributes of the indicator
    \item output: None                             
    \item exception: None
\end{itemize}
removeMsg():
\begin{itemize}
    \item transition: notify.text :=``"
    \item output: None
    \item exception: None
\end{itemize}
isValid(s):
\begin{itemize}
    \item transition: Check whether all the characters in s are in the pool
    \item output: True if all the characters in s are in the pool. False otherwise
    \item exception: None
\end{itemize}
isIN(target, pool):
\begin{itemize}
    \item transition: Find an element of pool that matches target
    \item output: True if that element exists. False otherwise
    \item exception: None
\end{itemize}
\subsubsection{Local Functions}
None
\newpage

\section{MIS of Pause Manager (\mref{Controller3})} 

\subsection{Module}
PauseManager
\subsection{Uses}
UnityEngine\\
UnityEngine.UI\\
\subsection{Syntax}

\subsubsection{Exported Constants}
None

\subsubsection{Exported Access Programs}
\begin{center}
\begin{tabular}{|p{3cm}|p{4cm}|p{4cm}| p{5cm}|}
\hline
\textbf{Name} & \textbf{In} & \textbf{Out} & \textbf{Exceptions} \\
\hline
 Start& &  & \\
 \hline
 Update& &  & \\
\hline
\end{tabular}
\end{center}

\subsection{Semantics}

\subsubsection{State Variables}
isPaused: Boolean\\
pauseIndicator: Image 
\subsubsection{Environment Variables}
None

\subsubsection{Assumptions}
None

\subsubsection{Access Routine Semantics}
Start():
\begin{itemize}
    \item transition: enable the visibility of the Pause image
    \item output: None
    \item exception: None
\end{itemize}
Update():
\begin{itemize}
    \item transition: Capture the event of pressing the K-key, change the value of isPaused, and enable or disable the visibility of the Pause image  
    \item output: None
    \item exception: None
\end{itemize}
\subsubsection{Local Functions}
None
\newpage

\section{MIS of Player Movement(\mref{Controller4})} \subsection{Module}
PlayerMovement
\subsection{Uses}
UnityEngine
\subsection{Syntax}

\subsubsection{Exported Constants}
None

\subsubsection{Exported Access Programs}
\begin{center}
\begin{tabular}{|p{3cm}|p{4cm}|p{4cm}| p{5cm}|}
\hline
\textbf{Name} & \textbf{In} & \textbf{Out} & \textbf{Exceptions} \\
\hline
 Update& &  & \\
\hline
\end{tabular}
\end{center}

\subsection{Semantics}
\subsubsection{State Variables}
controller: CharacterController
speed: $\mathbb{R}$

\subsubsection{Environment Variables}
None

\subsubsection{Assumptions}
None

\subsubsection{Access Routine Semantics}
\begin{itemize}
    \item transition: Change the position of the camera in each frame
    \item output: None
    \item exception: None
\end{itemize}
\subsubsection{Local Functions}
None
\newpage

\section{MIS of New Data Input Box Controller  (\mref{Controller5})} 

\subsection{Module}
NewDataInputBoxController
\subsection{Uses}
UnityEngine
\subsection{Syntax}

\subsubsection{Exported Constants}
None

\subsubsection{Exported Access Programs}
None

\subsection{Semantics}

\subsubsection{State Variables}
None

\subsubsection{Environment Variables}
None

\subsubsection{Assumptions}
None

\subsubsection{Access Routine Semantics}
None
\subsubsection{Local Functions}
None
\newpage

\section{MIS of Start Button Controller (\mref{Controller6})} 

\subsection{Module}
StartButtonController
\subsection{Uses}
UnityEngine
\subsection{Syntax}

\subsubsection{Exported Constants}
None
\subsubsection{Exported Access Programs}
\begin{center}
\begin{tabular}{|p{3cm}|p{4cm}|p{4cm}| p{5cm}|}
\hline
\textbf{Name} & \textbf{In} & \textbf{Out} & \textbf{Exceptions} \\
\hline
 OnClick& &  & \\
\hline
\end{tabular}
\end{center}
\subsection{Semantics}
\subsubsection{State Variables}
None

\subsubsection{Environment Variables}
None

\subsubsection{Assumptions}
None

\subsubsection{Access Routine Semantics}
OnClick():
\begin{itemize}
    \item transition: Enable the visibility of \mref{Viwer18} 
    \item output: None
    \item exception: None
\end{itemize}
\subsubsection{Local Functions}
None
\newpage

\section{MIS of Instruction Button Controller (\mref{Controller7})} 

\subsection{Module}
InstructionButtonController

\subsection{Uses}
None
\subsection{Syntax}

\subsubsection{Exported Constants}
None

\subsubsection{Exported Access Programs}
\begin{center}
\begin{tabular}{|p{3cm}|p{4cm}|p{4cm}| p{5cm}|}
\hline
\textbf{Name} & \textbf{In} & \textbf{Out} & \textbf{Exceptions} \\
\hline
 onClick& &  & \\
\hline
\end{tabular}
\end{center}

\subsection{Semantics}
\subsubsection{State Variables}
None
\subsubsection{Environment Variables}
None

\subsubsection{Assumptions}
None

\subsubsection{Access Routine Semantics}
onClick():
\begin{itemize}
    \item transition: Enable the visibility of  \mref{Viwer6}
    \item output: None
    \item exception: None
\end{itemize}
\subsubsection{Local Functions}
None
\newpage

\section{MIS of Update Data Button Controller (\mref{Controller8})} 

\subsection{Module}
UpdateDataButtonController
\subsection{Uses}
UnityEngine\\
UnityEngine.UI\\
Json File Reader\\
Value Indicator\\
\mref{Controller1}\\

\subsection{Syntax}

\subsubsection{Exported Constants}
None

\subsubsection{Exported Access Programs}
\begin{center}
\begin{tabular}{|p{6cm}|p{3cm}|p{3cm}| p{4cm}|}
\hline
\textbf{Name} & \textbf{In} & \textbf{Out} & \textbf{Exceptions} \\
\hline
 UpdateEnvDataButtonHandle& &  & \\
 UpdateTreeParamHande& &  & \\
 onClick & & &\\
\hline
\end{tabular}
\end{center}

\subsection{Semantics}
\subsubsection{State Variables}
 EnvDataOptions: Canvas\\
 TreeParamOptions: Canvas\\
 treeSelection: Dropdown\\
 indicator: ValueIndicator\\
 FR: FileReader

\subsubsection{Environment Variables}
None

\subsubsection{Assumptions}
None

\subsubsection{Access Routine Semantics}
UpdateEnvDataButtonHandle():
\begin{itemize}
\item transition: Change the isEnvData and isTreeParam variables in the indicator. 
\item output: None
\item exception: None
\end{itemize}
UpdateTreeParamHande():
\begin{itemize}
\item transition: Change the isTreeParam and isEnvData, in the indicator, and also change the TreeType attribute in the indicator according to the value of the Dropdown class. 
\item output: None
\item exception: None
\end{itemize}
onClick():
\begin{itemize}
\item transition: Invoke UpdateEnvDataButtonHandle() and UpdateTreeParamHande()
\item output: None
\item exception: None
\end{itemize}

\subsubsection{Local Functions}
None
\newpage

\section{MIS of Contact Us Button Controller (\mref{Controller9})} 

\subsection{Module}
ContactUsButtonController
\subsection{Uses}
UnityEngine
\subsection{Syntax}

\subsubsection{Exported Constants}
None

\subsubsection{Exported Access Programs}
\begin{center}
\begin{tabular}{|p{6cm}|p{3cm}|p{3cm}| p{4cm}|}
\hline
\textbf{Name} & \textbf{In} & \textbf{Out} & \textbf{Exceptions} \\
\hline
 onClick & & &\\
\hline
\end{tabular}
\end{center}

\subsection{Semantics}
\subsubsection{State Variables}
None

\subsubsection{Environment Variables}
None

\subsubsection{Assumptions}
None

\subsubsection{Access Routine Semantics}
onClick():
\begin{itemize}
    \item transition: Enable the visibility of  \mref{Viwer7}
    \item output: None
    \item exception: None
\end{itemize}
\subsubsection{Local Functions}
None
\newpage

\section{MIS of Quit Button Controller (\mref{Controller10})} 

\subsection{Module}
QuitButtonController
\subsection{Uses}
UnityEngine
\subsection{Syntax}

\subsubsection{Exported Constants}
None

\subsubsection{Exported Access Programs}
\begin{center}
\begin{tabular}{|p{3cm}|p{4cm}|p{4cm}| p{5cm}|}
\hline
\textbf{Name} & \textbf{In} & \textbf{Out} & \textbf{Exceptions} \\
\hline
 QuitSoftware & &  & \\
\hline
 onClick & &  & \\
\hline
\end{tabular}
\end{center}

\subsection{Semantics}
\subsubsection{State Variables}
None

\subsubsection{Environment Variables}
None

\subsubsection{Assumptions}
None

\subsubsection{Access Routine Semantics}
QuitSoftware():
\begin{itemize}
\item transition: Quit the software.
\item output: None
\item exception: None
\end{itemize}
onClick():
\begin{itemize}
\item transition: Invoke QuitSoftware()
\item output: None
\item exception: None
\end{itemize}
\subsubsection{Local Functions}
None
\newpage



\newpage

%%%%%%%%%%%%%%%%%%%%%%%%%%%%%%%%%%%%%%%%%%%%%%%
\section{MIS of Back Button Controller (\mref{Controller11})}  

\subsection{Module}
BackButtonController

\subsection{Uses}
UnityEngine\\
UnityEngine.SceneManagement\\

\subsection{Syntax}
\subsubsection{Exported Constants}
None
\subsubsection{Exported Access Programs}

\begin{center}
\begin{tabular}{|l|l|l|p{5cm}|}
\hline
\textbf{Name} & \textbf{In} & \textbf{Out} & \textbf{Exceptions} \\
\hline
onClick & mouse click &  &  \\
\hline
Back & &  & \\
\hline
\end{tabular}
\end{center}

\subsection{Semantics}

\subsubsection{State Variables}
viewState\\
upperLevelPage

\subsubsection{Environment Variables}
Mouse
\subsubsection{Assumptions}
None
\subsubsection{Access Routine Semantics}

\noindent Back():
\begin{itemize}
\item transition: upperLevelPage $\mathit{\implies}$(viewState := upperLevelPage)
\item output: None
\item exception: None
\end{itemize}

\subsubsection{Local Functions}
None

%%%%%%%%%%%%%%%%%%%%%%%%%%%%%%%%%%%%%%%%%%%%%%%

\newpage

%%%%%%%%%%%%%%%%%%%%%%%%%%%%%%%%%%%%%%%%%%%%%%%
\section{MIS of Plot Selection Drop Down Controller (\mref{Controller12})} 

\subsection{Module}
PlotSelectionDropDownController

\subsection{Uses}
UnityEngine.UI\\
UnityEngine.SceneManagement\\
System.Threading\\

\subsection{Syntax}
\subsubsection{Exported Constants}
None
\subsubsection{Exported Access Programs}

\begin{center}
\begin{tabular}{| l | l | l | p{5cm}|}
\hline
\textbf{Name} & \textbf{In} & \textbf{Out} & \textbf{Exceptions} \\
\hline
onClick & mouse click &  &  \\
\hline
displayMenu &&&\\
\hline
extractTreeParam & s: int &  &  \\
\hline
\end{tabular}
\end{center}

\subsection{Semantics}

\subsubsection{State Variables}
isActive: Boolean\\
s1: String\\
s2: String\\
s3: String\\
s4: String\\
s5: String\\
curIndex: int

\subsubsection{Environment Variables}
Mouse\\
DataModelObj: The gameobject of the current script\\
EnvDisp: Interface that will be displayed in Unity\\
dropDown: Drop down menu to select plot\\
\subsubsection{Assumptions}
None
\subsubsection{Access Routine Semantics}

\noindent displayMenu():
\begin{itemize}
\item transition: isActive$\mathit{:= \neg }$ isActive
\item output: None
\item exception: None
\end{itemize}


\noindent extractTreeParam(s):
\begin{itemize}
\item transition: Get the mouse click, assign different values to s1,s2,s3,s4,s5 based on the value of curIndex
\item output: None
\item exception: None
\end{itemize}



\subsubsection{Local Functions}
None

%%%%%%%%%%%%%%%%%%%%%%%%%%%%%%%%%%%%%%%%%%%%%%%

\newpage

%%%%%%%%%%%%%%%%%%%%%%%%%%%%%%%%%%%%%%%%%%%%%%%
\section{MIS of Tree Type Selection Drop Down Controller(\mref{Controller13})} 

\subsection{Module}

TreeTypeSelectionDropDownController

\subsection{Uses}
UnityEngine.UI\\
UnityEngine.SceneManagement\\
System.Threading\\

\subsection{Syntax}

\subsubsection{Exported Constants}
None
\subsubsection{Exported Access Programs}

\begin{center}
\begin{tabular}{| l | l | l | p{5cm}|}
\hline
\textbf{Name} & \textbf{In} & \textbf{Out} & \textbf{Exceptions} \\
\hline
onClick & mouse click  &   & \\
\hline
displayMenu &&&\\
\hline
extractTreeParam & s: int &  &  \\
\hline
\end{tabular}
\end{center}

\subsection{Semantics}

\subsubsection{State Variables}
isActive: Boolean\\
curIndex: int\\
s1: String\\
s2: String\\
s3: String\\
s4: String\\


\subsubsection{Environment Variables}
Mouse\\
DataModelObj: The gameobject of the current script\\
TreeParamDisp: Interface that will be displayed in Unity\\
dropdown: The drop down menu to select tree type
\subsubsection{Assumptions}
None
\subsubsection{Access Routine Semantics}

\noindent displayMenu():
\begin{itemize}
\item transition: isActive$\mathit{:= \neg }$ isActive
\item output: None
\item exception: None

\end{itemize}


\noindent extractTreeParam(s):
\begin{itemize}
\item transition: Get the mouse click, assign different values to s1,s2,s3,s4 based on the value of curIndex
\item output: None
\item exception: None
\end{itemize}

\subsubsection{Local Functions}
None

%%%%%%%%%%%%%%%%%%%%%%%%%%%%%%%%%%%%%%%%%%%%%

\newpage

%%%%%%%%%%%%%%%%%%%%%%%%%%%%%%%%%%%%%%%%%%%%%%%
\section{MIS of Show Environmental Data Button Controller (\mref{Controller14})} 

\subsection{Module}

ShowEnvDataButtoController

\subsection{Uses}
UnityEngine\\
UnityEngine.UI\\

\subsection{Syntax}

\subsubsection{Exported Constants}
None
\subsubsection{Exported Access Programs}

\begin{center}
\begin{tabular}{| l | l | l | p{5cm}|}
\hline
\textbf{Name} & \textbf{In} & \textbf{Out} & \textbf{Exceptions} \\
\hline
onClick & mouse click &  &  \\
\hline
EnvDataDispHandle &&&\\
\hline
\end{tabular}
\end{center}

\subsection{Semantics}

\subsubsection{State Variables}
displayEnvData: Boolean

\subsubsection{Environment Variables}
Mouse
\subsubsection{Assumptions}
None
\subsubsection{Access Routine Semantics}

\noindent EnvDataDispHandle():
\begin{itemize}
\item transition: displayEnvData $\mathit {:= \neg}$ displayEnvData
\item output: None
\item exception: None
\end{itemize}

\subsubsection{Local Functions}
None
%%%%%%%%%%%%%%%%%%%%%%%%%%%%%%%%%%%%%%%%%%%%%%%%%%%

\newpage

%%%%%%%%%%%%%%%%%%%%%%%%%%%%%%%%%%%%%%%%%%%%%%%%%%%
\section{MIS of Show Tree Parameter Button Controller(\mref{Controller15})}  

\subsection{Module}

ShowTreeParamButtonController

\subsection{Uses}
UnityEngine\\
UnityEngine.UI\\

\subsection{Syntax}

\subsubsection{Exported Constants}
None
\subsubsection{Exported Access Programs}

\begin{center}
\begin{tabular}{| l | l | l | p{5cm}|}
\hline
\textbf{Name} & \textbf{In} & \textbf{Out} & \textbf{Exceptions} \\
\hline
onClick & mouse click &  &  \\
\hline
TreeParamDispHandle &&&\\
\hline
\end{tabular}
\end{center}

\subsection{Semantics}

\subsubsection{State Variables}
isActive: Boolean

\subsubsection{Environment Variables}
Mouse
\subsubsection{Assumptions}
None
\subsubsection{Access Routine Semantics}

\noindent TreeParamDispHandle():
\begin{itemize}
\item transition: isActive $\mathit{:= \neg}$ isActive
\item output: None
\item exception: None
\end{itemize}
\subsubsection{Local Functions}
None

%%%%%%%%%%%%%%%%%%%%%%%%%%%%%%%%%%%%%%%%%%

\newpage

%%%%%%%%%%%%%%%%%%%%%%%%%%%%%%%%%%%%%%%%%%%%

\section{MIS of Environmental Selection Button Controller(\mref{Controller16})} 

\subsection{Module}
EnvDataSelectionButtonController


\subsection{Uses}
UnityEngine\\
UnityEngine.UI\\

\subsection{Syntax}

\subsubsection{Exported Constants}
None
\subsubsection{Exported Access Programs}

\begin{center}
\begin{tabular}{| l | l | l | p{5cm}|}
\hline
\textbf{Name} & \textbf{In} & \textbf{Out} & \textbf{Exceptions} \\
\hline
onClick & mouse click &  &  \\
\hline
displayEnvSel &&&\\
\hline
\end{tabular}
\end{center}

\subsection{Semantics}

\subsubsection{State Variables}
isActive: Boolean

\subsubsection{Environment Variables}
Mouse
\subsubsection{Assumptions}
None
\subsubsection{Access Routine Semantics}

\noindent displayEnvSel():
\begin{itemize}
\item transition: isActive $\mathit{:= \neg}$ isActive
\item output: None
\item exception: None
\end{itemize}
\subsubsection{Local Functions}
None

%%%%%%%%%%%%%%%%%%%%%%%%%%%%%%%%%%%%%%%%%%%%%%%%%

\newpage

%%%%%%%%%%%%%%%%%%%%%%%%%%%%%%%%%%%%%%%%%%%%%%%%%
\section{MIS of Data Type Selection Buttons Controller(\mref{Controller17})}  

\subsection{Module}
DataTypeSelectionButtonsController

\subsection{Uses}
UnityEngine\\
UnityEngine.UI\\

\subsection{Syntax}

\subsubsection{Exported Constants}
None
\subsubsection{Exported Access Programs}

\begin{center}
\begin{tabular}{| l | l | l | p{5cm}|}
\hline
\textbf{Name} & \textbf{In} & \textbf{Out} & \textbf{Exceptions} \\
\hline
onClick & mouse click &  &  \\
\hline
displayDataTypeSel &&&\\
\hline
\end{tabular}
\end{center}

\subsection{Semantics}

\subsubsection{State Variables}
isActive: Boolean

\subsubsection{Environment Variables}
Mouse
\subsubsection{Assumptions}
None
\subsubsection{Access Routine Semantics}

\noindent displayDataTypeSel():
\begin{itemize}
\item transition: isActive $\mathit{:= \neg}$ isActive
\item output: None
\item exception: None
\end{itemize}

\subsubsection{Local Functions}
None

%%%%%%%%%%%%%%%%%%%%%%%%%%%%%%%%%%%%%%%%%%%%%%%%%%%

\newpage

%%%%%%%%%%%%%%%%%%%%%%%%%%%%%%%%%%%%%%%%%%%%%%%%%%%%

\section{MIS of Save Button Controller(\mref{Controller18})}  

\subsection{Module}

SaveButtonController

\subsection{Uses}
UnityEngine\\
UnityEngine.UI\\
\mref{Controller19}

\subsection{Syntax}
\subsubsection{Exported Constants}
None
\subsubsection{Exported Access Programs}

\begin{center}
\begin{tabular}{| l | l | l | p{5cm}|}
\hline
\textbf{Name} & \textbf{In} & \textbf{Out} & \textbf{Exceptions} \\
\hline
onClick & mouse click &  &  \\
\hline
Save & updatedData: float &&\\
\hline
\end{tabular}
\end{center}

\subsection{Semantics}

\subsubsection{State Variables}
originalData: float\\
updatedData: float
\subsubsection{Environment Variables}
Mouse
\subsubsection{Assumptions}
None
\subsubsection{Access Routine Semantics}

\noindent Save():
\begin{itemize}
\item transition: originalData := updatedData
\item output: None
\item exception: None
\end{itemize}


\subsubsection{Local Functions}
None

%%%%%%%%%%%%%%%%%%%%%%%%%%%%%%%%%%%%%%%%%%%%%%%%

\newpage

%%%%%%%%%%%%%%%%%%%%%%%%%%%%%%%%%%%%%%%%%%%%%%%%%%%%

\section{MIS of File Writer(\mref{Controller19})}  

\subsection{Module}

FileWriter

\subsection{Uses}
UnityEngine\\
UnityEngine.UI\\
TMPro\\
System.IO\\
Newtonsoft.Json\\

\subsection{Syntax}
\subsubsection{Exported Constants}
None
\subsubsection{Exported Access Programs}

\begin{center}
\begin{tabular}{| l | l | l | p{5cm}|}
\hline
\textbf{Name} & \textbf{In} & \textbf{Out} & \textbf{Exceptions} \\
\hline
Awake &  &  &  \\
\hline
updateHandler &  & & InvalidInput\\
\hline
writeFile & inputText: String &&\\
\hline
\end{tabular}
\end{center}

\subsection{Semantics}

\subsubsection{State Variables}
indicator: Forest data type of the input\\
inputField: Textbox of the input\\\
inputText: input to the textbox\\
notify: text\\
FR: FileReader\\
JSON file\\

\subsubsection{Environment Variables}
None
\subsubsection{Assumptions}
None
\subsubsection{Access Routine Semantics}

\noindent Awake():
\begin{itemize}
\item transition: notify := NULL
\item output: None
\item exception: None
\end{itemize}

\noindent updateHandler():
\begin{itemize}
\item transition: notify := ``Updated"\\
inputField := NULL\\
Update the JSON file if the input is valid.\\
\item output: None\\
\item exception: $inputText == NULL \lor inputText \in \{(a,A), (z,Z)\}$
\end{itemize}

\noindent writeFile(inputText):
\begin{itemize}
\item transition: Use inputText to update the JSON files\\
\item output: None\\
\item exception: $inputText == NULL \lor inputText \in \{(a,A), (z,Z)\}$
\end{itemize}


\subsubsection{Local Functions}
\noindent removemsg():
\begin{itemize}
\item transition: notify := NULL\\
\item output: None\\
\item exception: None\\
\end{itemize}

\noindent isValid(s: string):
\begin{itemize}
\item transition: None\\
\item output: Boolean\\
\item exception: None\\
\end{itemize}

\noindent isIN(target: char, pool: char[]):
\begin{itemize}
\item transition: None\\
\item output: Boolean\\
\item exception: None\\
\end{itemize}

%%%%%%%%%%%%%%%%%%%%%%%%%%%%%%%%%%%%%%%%%%%%%%%%

\newpage

%%%%%%%%%%%%%%%%%%%%%%%%%%%%%%%%%%%%%%%%%%%%%%%%%%%%

\section{MIS of File Reader(\mref{Controller20})}  

\subsection{Module}

FileWriter

\subsection{Uses}
UnityEngine\\
UnityEngine.UI\\
TMPro\\
System.IO\\
Newtonsoft.Json\\

\subsection{Syntax}
\subsubsection{Exported Constants}
None
\subsubsection{Exported Access Programs}

\begin{center}
\begin{tabular}{| l | l | l | p{5cm}|}
\hline
\textbf{Name} & \textbf{In} & \textbf{Out} & \textbf{Exceptions} \\
\hline
readEnvData &  &  &  \\
\hline
readTreeParam &  & & \\
\hline
clearText &&&\\
\hline
\end{tabular}
\end{center}

\subsection{Semantics}

\subsubsection{State Variables}
indicator: Forest data type of the input\\
JsonModelObj: \\
currentValueDisp: Output shown
JSON file\\

\subsubsection{Environment Variables}
None
\subsubsection{Assumptions}
None
\subsubsection{Access Routine Semantics}

\noindent readEnvData():
\begin{itemize}
\item transition: filePath := "./dataCenter/plot" + plotNumber.ToString() + "data.json";\\
plotJsonData := File.ReadAllText(filePath);\\
JsonModelObj := Newtonsoft.Json.JsonConvert.DeserializeObject<JsonModel>(plotJsonData);\\
EnvDataType := indicator.EnvDataType;\\
currentValueDisp.text := currentValueDisp.text + JasonModelObj.envDataType.EnvDataType\\
\item output: None
\item exception: None
\end{itemize}

\noindent readTreeParam():
\begin{itemize}
\item transition: filePath := "./dataCenter/plot" + plotNumber.ToString() + "data.json";\\
plotJsonData := File.ReadAllText(filePath);\\
JsonModelObj := Newtonsoft.Json.JsonConvert.DeserializeObject<JsonModel>(plotJsonData);\\
TreeType := indicator.TreeType;\\
TreeParamType := indicator.TreeParamType;\\
currentValueDisp.text := currentValueDisp.text + JasonModelObj.TreeType.TreeParamType\\
\item output: None
\item exception: None
\end{itemize}

\noindent clearText():
\begin{itemize}
\item transition:  currentValueDisp.text := None;
\item output: None
\item exception: None
\end{itemize}

\subsubsection{Local Functions}
None

%%%%%%%%%%%%%%%%%%%%%%%%%%%%%%%%%%%%%%%%%%%%%%%%

\newpage

%%%%%%%%%%%%%%%%%%%%%%%%%%%%%%%%%%%%%%%%%%%%%%%%%%%%

\section{MIS of Season Change Controller(\mref{Controller21})}  

\subsection{Module}

SeasonChangeController

\subsection{Uses}
UnityEngine\\
UnityEngine.UI\\
treePlantingController\\
SeasonChangeButtonController

\subsection{Syntax}
\subsubsection{Exported Constants}
None
\subsubsection{Exported Access Programs}

\begin{center}
\begin{tabular}{| l | l | l | p{5cm}|}
\hline
\textbf{Name} & \textbf{In} & \textbf{Out} & \textbf{Exceptions} \\
\hline
Awake &  &  &  \\
\hline
changeSeason &  & & \\
\hline
\end{tabular}
\end{center}

\subsection{Semantics}

\subsubsection{State Variables}
haveLeaves: If the trees have leaves\\
seasonChangeBut: Season Change button\\
Snow: Particle System for snowing effects

\subsubsection{Environment Variables}
None
\subsubsection{Assumptions}
None
\subsubsection{Access Routine Semantics}

\noindent Awake():
\begin{itemize}
\item transition: seasonChangeBut.image.sprite := summerImage;\\
        Snow.gameObject.SetActive(false)
\item output: None
\item exception: None
\end{itemize}

\noindent changeSeason():
\begin{itemize}
\item transition: haveLeaves := !haveLeaves;\\
(haveLeaves $\implies$ Snow.gameObject.SetActive(false)$\land$
             (seasonChangeBut.image.sprite := summerImage));\\
(!haveLeaves $\implies$Snow.gameObject.SetActive(true)$\land$(seasonChangeBut.image.sprite := winterImage))\\
\item output: None
\item exception: None
\end{itemize}




\subsubsection{Local Functions}
None
%%%%%%%%%%%%%%%%%%%%%%%%%%%%%%%%%%%%%%%%%%%%%%%%


%%%%%%%%%%%%%%%%%%%%%%%%%%%%%%%%%%%%%%%%%%%%%%%%

\newpage


\section{MIS of move Panel Controller(\mref{Controller22})}  

\subsection{Module}
movePanelController


\subsection{Uses}

\subsection{Syntax}
\subsubsection{Exported Constants}
None
\subsubsection{Exported Access Programs}

\begin{center}
\begin{tabular}{| l | l | l | p{5cm}|}
\hline
\textbf{Name} & \textbf{In} & \textbf{Out} & \textbf{Exceptions} \\
\hline
 myClick&  &  &  \\
\hline
 Update&  & & \\
\hline
\end{tabular}
\end{center}

\subsection{Semantics}

\subsubsection{State Variables}


\subsubsection{Environment Variables}
float speed = 2000f;\\
float speed2 = -2000f;\\
bool isActive = false;\\
bool isShown = false;\\
bool isActive2 = false;\\
bool isShown2 = false;\\
int l\_boundary = -400;\\
int r\_boundary = 350;\\
int l\_boundary2 = 3100;\\
int r\_boundary2 = 3850;\\
GameObject go;\\
GameObject go2;
\subsubsection{Assumptions}
None
\subsubsection{Access Routine Semantics}

\noindent myClick():
\begin{itemize}
\item transition: (atBoundary()$\implies$(isActive := true $\land $ isShown := false))\\$\land$(!atBoundary()$\implies$(isActive := false $\land $ isShown := true))
\item output: None
\item exception: None
\end{itemize}

\noindent Update():
\begin{itemize}
\item transition: 
((isActive $\land$ !isShown) $\implies$ go.transform.Translate(speed * Time.deltaTime, 0, 0) $\land$
((go.transform.position.x $\leqslant$ r$\_$boundary) $\implies$ (isShown:=true)))\\

((!isActive $\land$ !isShown) $\implies$  go.transform.Translate(-speed * Time.deltaTime, 0, 0); $\land$
((go.transform.position.x $\leqslant$ l$\_$boundary) $\implies$ (isShown:=true)))\\

((isActive2 $\land$ !isShown2) $\implies$ go2.transform.Translate(speed2 * Time.deltaTime, 0, 0) $\land$
((go2.transform.position.x $\leqslant$ l$\_$boundary2) $\implies$ (isShown:=true)))\\

((!isActive2 $\land$ !isShown2) $\implies$  go2.transform.Translate(-speed2 * Time.deltaTime, 0, 0); $\land$
((go2.transform.position.x $\leqslant$ r$\_$boundary2) $\implies$ (isShown:=true)))

\item output: None
\item exception: None
\end{itemize}



\subsubsection{Local Functions}

\noindent atBoundary():
\begin{itemize}
\item transition: x := go.transform.position.x\\
\item output: x$\leqslant$ l\_boundary\\
\item exception: None\\
\end{itemize}
%%%%%%%%%%%%%%%%%%%%%%%%%%%%%%%%%%%%%%%%%%%%%%%%

\newpage

\section{MIS of Tree Planting Controller(\mref{Controller23})}  

\subsection{Module}

treePlantingController

\subsection{Uses}
UnityEngine\\
UnityEngine.UI\\
seasonController\\

\subsection{Syntax}
\subsubsection{Exported Constants}
None
\subsubsection{Exported Access Programs}

\begin{center}
\begin{tabular}{| l | l | l | p{5cm}|}
\hline
\textbf{Name} & \textbf{In} & \textbf{Out} & \textbf{Exceptions} \\
\hline
 plantTrees& plotNumberIndex  &  &  \\
\hline
 cleanTrees&  & & \\
\hline
\end{tabular}
\end{center}

\subsection{Semantics}

\subsubsection{State Variables}


\subsubsection{Environment Variables}
\textbf{Related measures in unity tree editor:}

\noindent double redPineHSR = 25.0;\\
double oakHSR = 17.2 / 2;\\
double beechHSR = 35.47 / 2;\\
double birchHSR = 22.18 / 2;\\
double redMapleHSR = 14.12 / 2;   \\
double whitePineHSR = 64.4 / 2;\\
double redOakHSR = 18.95 / 2\\

\noindent\textbf{Plot Information}:\\
double sideLength = 100.0;\\
float startingCoordinate = 1f;\\
float endingCoordinate = 99f\\

\noindent\textbf{Collection of tree positions:}\\
Vector3[] treelocal\\

\noindent\textbf{Treefabs with leaves:}\\
List$<GameObject>$treeprefabsWL\\

\noindent\textbf{Treefabs without leaves:}\\
List$<GameObject>$ treeprefabsWOL\\

\noindent \textbf{ListOfCircles}
circles[plotNumberIndex]\\


\subsubsection{Assumptions}
None
\subsubsection{Access Routine Semantics}

\noindent plantTrees(plotNumberIndex):
\begin{itemize}
\item transition: generateCircleLocation(circles(plotNumberIndex),differentTreeNumbers.Sum());\\
(seasonController.haceLeaves $\implies$ treeprefabWL);\\
(!seasonController.haveLeaves $\implies$ treeprefabWOL);\\
($\forall$ species $\implies$ standardScale := data.species.Height/HSR $\land$ lowerBound := 0.9 * standardScale $\land$ upperBound := 1.1 * standardScale);\\
($\forall$ species $\implies$ trees.Add(Instantiate(treeprefabs $|$ treeinstance.transform.localscale = Vectror3(randomScale, randomScale, randomScale)))\\

\item output: None
\item exception: None
\end{itemize}

\noindent cleanTrees():
\begin{itemize}
\item transition: $\forall$ tree $\in$ trees $\implies$ Destroy(tree);
\item output: None
\item exception: None
\end{itemize}



\subsubsection{Local Functions}
\noindent calculatedTreeNumbers():
\begin{itemize}
\item transition: area := sideLength *sideLength;\\
(int) treenumbers =$\sum_{species}$ density*area

\item output: treenumbers
\item exception: None
\end{itemize}

\noindent generateSquareLocation(numberOfPoints):
\begin{itemize}
\item transition: 
treelocal[i] = Vector3(UnityEngine.Random.Range(startingCoordinate, endingCoordinate), 0f, UnityEngine.Random.Range(startingCoordinate, endingCoordinate))
\item output: None
\item exception: None
\end{itemize}


\noindent generateCirculeLocation(List<Circle> circles, int numberOfPoints):
\begin{itemize}
\item transition: 
Point randomPoint := Point(UnityEngine.Random.Range(startingCoordinate, endingCoordinate), UnityEngine.Random.Range(startingCoordinate, endingCoordinate));\\
(isPointInCircle $\implies$ treelocal[i] = Vector3((float)randomPoint.getX(),0f, (float) randomPoint.getY()));

\item output: None
\item exception: None
\end{itemize}

\noindent isPointInCircles(List$<Circle>$ circles, Point point):
\begin{itemize}
\item transition: None

\item output: ($\exists$ circle $\in$ circles $|$ circle.isIn(point) = true)

\item exception: None
\end{itemize}
%%%%%%%%%%%%%%%%%%%%%%%%%%%%%%%%%%%%%%%%%%%%%%%%

\newpage

\section{MIS of Tree Switch Button Controller(\mref{Controller24})}   

\subsection{Module}

TreeSwitchButtonController

\subsection{Uses}

UnityEngine\\
UnityEngine.UI\\

\subsection{Syntax}
\subsubsection{Exported Constants}
None
\subsubsection{Exported Access Programs}

\begin{center}
\begin{tabular}{| l | l | l | p{5cm}|}
\hline
\textbf{Name} & \textbf{In} & \textbf{Out} & \textbf{Exceptions} \\
\hline
 onClick&  &  &  \\
\hline
 
\end{tabular}
\end{center}

\subsection{Semantics}

\subsubsection{State Variables}
isActive: Boolean value indicating if the leaf information page is active. 


\subsubsection{Environment Variables}
None
\subsubsection{Assumptions}
None
\subsubsection{Access Routine Semantics}

\noindent onClick():
\begin{itemize}
\item transition: isActive := !isActive
\item output: None
\item exception: None
\end{itemize}


\subsubsection{Local Functions}
None
%%%%%%%%%%%%%%%%%%%%%%%%%%%%%%%%%%%%%%%%%%%%%%%%

\newpage

\section{MIS of Pie Chart Button Controller(\mref{Controller25})}  

\subsection{Module}

pieChartButtonController

\subsection{Uses}
UnityEngine\\
UnityEngine.UI\\
\subsection{Syntax}
\subsubsection{Exported Constants}
None
\subsubsection{Exported Access Programs}

\begin{center}
\begin{tabular}{| l | l | l | p{5cm}|}
\hline
\textbf{Name} & \textbf{In} & \textbf{Out} & \textbf{Exceptions} \\
\hline
onClick &  &  &  \\
\hline
\end{tabular}
\end{center}

\subsection{Semantics}

\subsubsection{State Variables}
isActive: Boolean value indicating if the pie chart is active. 

\subsubsection{Environment Variables}
None
\subsubsection{Assumptions}
None
\subsubsection{Access Routine Semantics}

\noindent onClick():
\begin{itemize}
\item transition: isActive := !isActive
\item output: None
\item exception: None
\end{itemize}


\subsubsection{Local Functions}
None
%%%%%%%%%%%%%%%%%%%%%%%%%%%%%%%%%%%%%%%%%%%%%%%%

\newpage

\section{MIS of Pie Chart Controller(\mref{Controller26})}  

\subsection{Module}

pieChartController

\subsection{Uses}
pieChartButtonController\\
UnityEngine\\
UnityEngine.UI
\subsection{Syntax}
\subsubsection{Exported Constants}
None
\subsubsection{Exported Access Programs}

\begin{center}
\begin{tabular}{| l | l | l | p{5cm}|}
\hline
\textbf{Name} & \textbf{In} & \textbf{Out} & \textbf{Exceptions} \\
\hline
 ChangeView&  &  &  \\
\hline
 markChart&  & & \\
\hline
\end{tabular}
\end{center}

\subsection{Semantics}

\subsubsection{State Variables}

isActive: Boolean value indicating if the pie chart is active. 

\subsubsection{Environment Variables}

GameObject graph\\
Image legend\\
EnvDataDisp\\

\noindent\textbf{Data and Image for generating pieChart:}\\
DataModel data\\
Image regPinePortion\\
Image oakPortion\\
Image beechPortion\\
Image birchPortion\\
Image redMaplePortion\\
Image whitePinePortion\\
Image redOakPortion\\

\noindent GameObject parentObject\\

\subsubsection{Assumptions}
None
\subsubsection{Access Routine Semantics}

\noindent ChangeView():
\begin{itemize}
\item transition: graph.gameObejct.SetActive(!isActive) $\land$ legend.gameObject.SetActive(!isActive) $\land$ EnvDataDisp.gameObject.SetActive(isActive);\\
isActive := !isActive
\item output: None
\item exception: None
\end{itemize}

\noindent markChart():
\begin{itemize}
\item transition: calculateValues();\\
setImagePortions()
\item output: None
\item exception: None
\end{itemize}



\subsubsection{Local Functions}
\noindent calculateValues():
\begin{itemize}
\item transition: percentage$_{species}$ := number$_{species}$ / $\sum_{all species}$ number
\item output: None
\item exception: None
\end{itemize}

\noindent setImagePortions():
\begin{itemize}
\item transition: pos := Vector3(150f,200f,0f);\\
size := Vector2(600f, 600f);\\
$\forall$ species $\implies$ (Portion$_{species}$.transform.locakPosition := pos \\
$\land$ Portion$_{species}$.rectTransform.sizeData := size \\
$\land$ Portion$_{species}$.fillAmount:= percentage$_{species}$ \\
$\land$ Portion$_{species}$.transform.rotation := Quaternion.Euler (Vector3 (0f, 0f, totalRot)) \\
$\land$ totalRot := totalRot(percentage$_{species}$))
\item output: None
\item exception: None
\end{itemize}


%%%%%%%%%%%%%%%%%%%%%%%%%%%%%%%%%%%%%%%%%%%%%%%%

\newpage

\section{MIS of Season Change Button Controller(\mref{Controller27})}  

\subsection{Module}

SeasonChangeButtonController

\subsection{Uses}

\subsection{Syntax}
\subsubsection{Exported Constants}
None
\subsubsection{Exported Access Programs}

\begin{center}
\begin{tabular}{| l | l | l | p{5cm}|}
\hline
\textbf{Name} & \textbf{In} & \textbf{Out} & \textbf{Exceptions} \\
\hline
 onClick&  &  &  \\

\hline
\end{tabular}
\end{center}

\subsection{Semantics}

\subsubsection{State Variables}
isActive: Boolean value indicating if the current season is summer.

\subsubsection{Environment Variables}
None
\subsubsection{Assumptions}
None
\subsubsection{Access Routine Semantics}

\noindent onClick():
\begin{itemize}
\item transition: isActive := !isActive
\item output: None
\item exception: None
\end{itemize}


\subsubsection{Local Functions}
None
%%%%%%%%%%%%%%%%%%%%%%%%%%%%%%%%%%%%%%%%%%%%%%%%

\newpage

\section{MIS of Tree Switch Controller(\mref{Controller28})}  

\subsection{Module}
TreeSwitchController


\subsection{Uses}
UnityEngine\\
UnityEngine.UI\\
TreeSwitchButtonController

\subsection{Syntax}
\subsubsection{Exported Constants}
None
\subsubsection{Exported Access Programs}

\begin{center}
\begin{tabular}{| l | l | l | p{5cm}|}
\hline
\textbf{Name} & \textbf{In} & \textbf{Out} & \textbf{Exceptions} \\
\hline
 ChangeView&  &  &  \\
\hline
\end{tabular}
\end{center}

\subsection{Semantics}

\subsubsection{State Variables}
isActive: Boolean value indicating if leaf information is active. \\
TreeParamDisplay: UI of Tree parameters display.\\
LeafInfoDisplay: UI of Leaf information display.

\subsubsection{Environment Variables}
None
\subsubsection{Assumptions}
None
\subsubsection{Access Routine Semantics}

\noindent ChangeView():
\begin{itemize}
\item transition: (isActive $\implies$ (LeafInfoDisplay.gameObject.SetActive(false) $\land$ TreeParamDisplay.gameObject.SetActive(true);\\

(!isActive $\implies$ (LeafInfoDisplay.gameObject.SetActive(true) $\land$ TreeParamDisplay.gameObject.SetActive(false);\\

isActive := !isActive
\item output: None
\item exception: None
\end{itemize}


\subsubsection{Local Functions}
None
%%%%%%%%%%%%%%%%%%%%%%%%%%%%%%%%%%%%%%%%%%%%%%%%

\newpage

%%%%%%%%%%%%%% TS Starts %%%%%%%%%%%%%%%%%%%%%%%%%%%%%%%%%%%%%%
%%%%%%%%%%% Forest Trees starts %%%%%%%%%%%%%%%%%%
\section{MIS of Forest Trees (\mref{Model1})} 

\subsection{Module}
ForestTrees

\subsection{Uses}
\mref{Model4} , \mref{Model5}, 
\mref{Model6}, \mref{Model7}, 
\mref{Model8}, \mref{Model9},
\mref{Model10}

\subsection{Syntax}

\subsubsection{Exported Constants}
None

\subsubsection{Exported Access Programs}
\begin{center}
\begin{tabular}{|p{3cm}|p{4cm}|p{4cm}| p{5cm}|}
\hline
\textbf{Name} & \textbf{In} & \textbf{Out} & \textbf{Exceptions} \\
\hline
new ForestTrees & & ForestTree & \\
\hline
addTree & GameObject(This is unity built-in type) & & \\ 
\hline
deleteTree & & & \\
\hline
\end{tabular}
\end{center}

\subsection{Semantics}

\subsubsection{State Variables}
$\mathit{trees: GameObject\{\} }$

\subsubsection{Environment Variables}
None

\subsubsection{Assumptions}
None

\subsubsection{Access Routine Semantics}
\noindent new ForestTrees():
\begin{itemize}
\item transition: None
\item output: $\mathit{out := self}$
\item exception: None
\end{itemize}

\noindent addTree(tree):
\begin{itemize}
\item transition: $\mathit{trees := trees } \cup tree $ 
\item output: None
\item exception: None
\end{itemize}

\noindent DeleteTree(s):
\begin{itemize}
\item transition: $\forall tree: GameObject | tree \in trees : tree.destory()$
\item output: None
\item exception: None
\end{itemize}

\subsubsection{Local Functions}
None
%%%%%%%%%%%%%%%%%%%%%%%%% Forest Trees ends %%%%%%%%%%%%%

\newpage

%%%%%%%%%%% Forest Sky starts %%%%%%%%%%%%%%%%%%
\section{MIS of Forest Sky (\mref{Model2})} 
\subsection{Module}

SkyBox

\subsection{Uses}

UnityLightning

\subsection{Syntax}

\subsubsection{Exported Constants}
None
\subsubsection{Exported Access Programs}


\begin{center}
\begin{tabular}{|p{3cm}|p{4cm}|p{4cm}| p{5cm}|}
\hline
\textbf{Name} & \textbf{In} & \textbf{Out} & \textbf{Exceptions} \\
\hline
new ForestSky & & ForestSky & \\
\hline
setSkyBox & Unity Texture & & \\ 
\hline
\end{tabular}
\end{center}

\subsection{Semantics}

\subsubsection{State Variables}
None

\subsubsection{Environment Variables}
SkyTexture: imported picture of the skybox.

\subsubsection{Assumptions}
Unity only takes valid texture file type as input.

\subsubsection{Access Routine Semantics}
\noindent new ForestSky()
\begin{itemize}
\item transition: None
\item output: None
\item exception: None
\end{itemize}

\noindent SetSkybox(s):
\begin{itemize}
\item transition: set the current skybox to the selected texture file.
\item output: None
\item exception: None
\end{itemize}

\subsubsection{Local Functions}
None
%%%%%%%%%%%%%%%%%%%%%%%%% Forest Sky ends %%%%%%%%%%%%%

\newpage

%%%%%%%%%%% Forest Terrain starts %%%%%%%%%%%%%%%%%%
\section{MIS of Forest Terrain (\mref{Model3})} 
\subsection{Module}
ForestTerrain

\subsection{Uses}
Unity Terrain Tool

\subsection{Syntax}

\subsubsection{Exported Constants}
None
\subsubsection{Exported Access Programs}

\begin{center}
\begin{tabular}{|p{3cm}|p{4cm}|p{4cm}| p{5cm}|}
\hline
\textbf{Name} & \textbf{In} & \textbf{Out} & \textbf{Exceptions} \\
\hline
new ForestTerrain & & ForestTerrain & \\
\hline
setLength & Double & & \\ 
\hline
setWidth & Double & & \\ 
\hline
\end{tabular}
\end{center}

\subsection{Semantics}

\subsubsection{State Variables}
None

\subsubsection{Environment Variables}
None

\subsubsection{Assumptions}
None

\subsubsection{Access Routine Semantics}

\noindent new ForestTerrain():
\begin{itemize}
\item transition: Create a new terrain in unity using the terrain tool
\item output: None
\item exception: None
\end{itemize}

\noindent setLenght(x):
\begin{itemize}
\item transition: set the length of the terrain to be x meters
\item output: None
\item exception: None
\end{itemize}

\noindent setWidth(x):
\begin{itemize}
\item transition: set the width of the terrain to be x meters
\item output: None
\item exception: None
\end{itemize}

\subsubsection{Local Functions}
None
%%%%%%%%%%%%%%%%%%%%%%%%% Forest Terrain ends %%%%%%%%%%%%%

\newpage

%%%%%%%%%%%%%%%%%%% RedPine starts %%%%%%%%%%%
\noindent 
\textcolor{red}{
We want to address the following two points regarding creating 
different modules for different tree types:
\begin{itemize}
\item We created different modules for different types of 
trees because this is necessary when it comes to reading from 
JSON files by using the newtonsoft parser.
\item Also, we deleted
the ``isValidString'' local function here since we will check the validity 
of the string when users input from the GUI
\end{itemize}}


\newcommand{\tn}{Red Pine }
\newcommand{\tmn}{RedPine}
\newcommand{\constn}{Red\ Pine}

\section{MIS of \tn (\mref{Model4})}

\subsection{Module}
\tmn

\subsection{Uses}
None

\subsection{Syntax}
\subsubsection{Exported Constants}
None
\subsubsection{Exported Access Programs}

\begin{center}
\begin{tabular}{|l|l|l| p{5cm}|}
\hline
\textbf{Name} & \textbf{In} & \textbf{Out} & \textbf{Exceptions} \\
\hline
new \tmn & & \tmn & \\
\hline
getTreeName & &String & \\
\hline 
setDensity & String & & \\
\hline
getDensity & & String & \\
\hline
setDBH & String & & \\
\hline
getDBH & & String & \\
\hline
setHeight & String & & \\
\hline
getHeight & & String & \\
\hline
setAge & String & & \\
\hline
getAge & & String & \\
\hline
\end{tabular}
\end{center}

\subsection{Semantics}

\subsubsection{State Variables}
$\mathit{Treename: String}$\\
$\mathit{Density: String}$\\
$\mathit{DBH: String}$\\
$\mathit{Height: String}$\\
$\mathit{Age: String}$\\

\subsubsection{Environment Variables}
None

\subsubsection{Assumptions}
None

\subsubsection{Access Routine Semantics}

\noindent new \tmn():
\begin{itemize}
\item transition: $\mathit{Treename, Density, DBH, Height,
Age := ``\constn", ``", ``", ``", ``"}$
\item output: $\mathit{out := self}$
\item exception: None 
\end{itemize}
\noindent getTreeName():
\begin{itemize}
\item transition: None
\item output: $\mathit{out := Treename}$
\item exception: None
\end{itemize}
\noindent setDensity(newDensity):
\begin{itemize}
\item transition: $\mathit{Density := newDensity}$
\item output: None
\item exception: None
\end{itemize}
\noindent getDensity():
\begin{itemize}
\item transition: None
\item output: $\mathit{out := Density}$
\item exception: None
\end{itemize}
\noindent setDBH(newDBH):
\begin{itemize}
\item transition: $\mathit{DBH := newDBH}$
\item output: None
\item exception: None
\end{itemize}
\noindent getDBH():
\begin{itemize}
\item transition: None
\item output: $\mathit{out := DBH}$
\item exception: None
\end{itemize}
\noindent setHeight(newHeight):
\begin{itemize}
\item transition: $\mathit{Height := newHeight}$
\item output: None
\item exception: None
\end{itemize}
\noindent getHeight():
\begin{itemize}
\item transition: None
\item output: $\mathit{out := Height}$
\item exception: None
\end{itemize}
\noindent setAge(newAge):
\begin{itemize}
\item transition: $\mathit{Age := newAge}$
\item output: None
\item exception: None
\end{itemize}
\noindent getAge():
\begin{itemize}
\item transition: None
\item output: $\mathit{out := Age}$
\item exception: None
\end{itemize}

\subsubsection{Local Functions}
None
 %%%%%%%%%%%%%%%%%%%%% Red Pine ends %%%%%%%%%%%%%

\newpage

%%%%%%%%%%%%%%%%%%% Oak starts %%%%%%%%%%%
\renewcommand{\tn}{Oak }
\renewcommand{\tmn}{Oak}
\renewcommand{\constn}{Oak}

\section{MIS of \tn (\mref{Model5})}

\subsection{Module}
\tmn

\subsection{Uses}
None

\subsection{Syntax}
\subsubsection{Exported Constants}
None
\subsubsection{Exported Access Programs}

\begin{center}
\begin{tabular}{|l|l|l| p{5cm}|}
\hline
\textbf{Name} & \textbf{In} & \textbf{Out} & \textbf{Exceptions} \\
\hline
new \tmn & & \tmn & \\
\hline
getTreeName & &String & \\
\hline 
setDensity & String & & \\
\hline
getDensity & & String & \\
\hline
setDBH & String & & \\
\hline
getDBH & & String & \\
\hline
setHeight & String & & \\
\hline
getHeight & & String & \\
\hline
setAge & String & & \\
\hline
getAge & & String & \\
\hline
\end{tabular}
\end{center}

\subsection{Semantics}

\subsubsection{State Variables}
$\mathit{Treename: String}$\\
$\mathit{Density: String}$\\
$\mathit{DBH: String}$\\
$\mathit{Height: String}$\\
$\mathit{Age: String}$\\

\subsubsection{Environment Variables}
None

\subsubsection{Assumptions}
None

\subsubsection{Access Routine Semantics}

\noindent new \tmn():
\begin{itemize}
\item transition: $\mathit{Treename, Density, DBH, Height,
Age := ``\constn", ``", ``", ``", ``"}$
\item output: $\mathit{out := self}$
\item exception: None 
\end{itemize}
\noindent getTreeName():
\begin{itemize}
\item transition: None
\item output: $\mathit{out := Treename}$
\item exception: None
\end{itemize}
\noindent setDensity(newDensity):
\begin{itemize}
\item transition: $\mathit{Density := newDensity}$
\item output: None
\item exception: None
\end{itemize}
\noindent getDensity():
\begin{itemize}
\item transition: None
\item output: $\mathit{out := Density}$
\item exception: None
\end{itemize}
\noindent setDBH(newDBH):
\begin{itemize}
\item transition: $\mathit{DBH := newDBH}$
\item output: None
\item exception: None
\end{itemize}
\noindent getDBH():
\begin{itemize}
\item transition: None
\item output: $\mathit{out := DBH}$
\item exception: None
\end{itemize}
\noindent setHeight(newHeight):
\begin{itemize}
\item transition: $\mathit{Height := newHeight}$
\item output: None
\item exception: None
\end{itemize}
\noindent getHeight():
\begin{itemize}
\item transition: None
\item output: $\mathit{out := Height}$
\item exception: None
\end{itemize}
\noindent setAge(newAge):
\begin{itemize}
\item transition: $\mathit{Age := newAge}$
\item output: None
\item exception: None
\end{itemize}
\noindent getAge():
\begin{itemize}
\item transition: None
\item output: $\mathit{out := Age}$
\item exception: None
\end{itemize}

\subsubsection{Local Functions}
None
 %%%%%%%%%%%%%%%%%%%%% Oak ends %%%%%%%%%%%%%

\newpage

%%%%%%%%%%%%%%%%%%% Beech starts %%%%%%%%%%%
\renewcommand{\tn}{Beech }
\renewcommand{\tmn}{Beech}
\renewcommand{\constn}{Beech}

\section{MIS of \tn (\mref{Model6})}

\subsection{Module}
\tmn

\subsection{Uses}
None

\subsection{Syntax}
\subsubsection{Exported Constants}
None
\subsubsection{Exported Access Programs}

\begin{center}
\begin{tabular}{|l|l|l| p{5cm}|}
\hline
\textbf{Name} & \textbf{In} & \textbf{Out} & \textbf{Exceptions} \\
\hline
new \tmn & & \tmn & \\
\hline
getTreeName & &String & \\
\hline 
setDensity & String & & \\
\hline
getDensity & & String & \\
\hline
setDBH & String & & \\
\hline
getDBH & & String & \\
\hline
setHeight & String & & \\
\hline
getHeight & & String & \\
\hline
setAge & String & & \\
\hline
getAge & & String & \\
\hline
\end{tabular}
\end{center}

\subsection{Semantics}

\subsubsection{State Variables}
$\mathit{Treename: String}$\\
$\mathit{Density: String}$\\
$\mathit{DBH: String}$\\
$\mathit{Height: String}$\\
$\mathit{Age: String}$\\

\subsubsection{Environment Variables}
None

\subsubsection{Assumptions}
None

\subsubsection{Access Routine Semantics}

\noindent new \tmn():
\begin{itemize}
\item transition: $\mathit{Treename, Density, DBH, Height,
Age := ``\constn", ``", ``", ``", ``"}$
\item output: $\mathit{out := self}$
\item exception: None 
\end{itemize}
\noindent getTreeName():
\begin{itemize}
\item transition: None
\item output: $\mathit{out := Treename}$
\item exception: None
\end{itemize}
\noindent setDensity(newDensity):
\begin{itemize}
\item transition: $\mathit{Density := newDensity}$
\item output: None
\item exception: None
\end{itemize}
\noindent getDensity():
\begin{itemize}
\item transition: None
\item output: $\mathit{out := Density}$
\item exception: None
\end{itemize}
\noindent setDBH(newDBH):
\begin{itemize}
\item transition: $\mathit{DBH := newDBH}$
\item output: None
\item exception: None
\end{itemize}
\noindent getDBH():
\begin{itemize}
\item transition: None
\item output: $\mathit{out := DBH}$
\item exception: None
\end{itemize}
\noindent setHeight(newHeight):
\begin{itemize}
\item transition: $\mathit{Height := newHeight}$
\item output: None
\item exception: None
\end{itemize}
\noindent getHeight():
\begin{itemize}
\item transition: None
\item output: $\mathit{out := Height}$
\item exception: None
\end{itemize}
\noindent setAge(newAge):
\begin{itemize}
\item transition: $\mathit{Age := newAge}$
\item output: None
\item exception: None
\end{itemize}
\noindent getAge():
\begin{itemize}
\item transition: None
\item output: $\mathit{out := Age}$
\item exception: None
\end{itemize}

\subsubsection{Local Functions}
None
 %%%%%%%%%%%%%%%%%%%%% Beech ends %%%%%%%%%%%%%

\newpage

%%%%%%%%%%%%%%%%%%% Birch starts %%%%%%%%%%%
\renewcommand{\tn}{Birch }
\renewcommand{\tmn}{Birch}
\renewcommand{\constn}{Birch}

\section{MIS of \tn (\mref{Model7})}

\subsection{Module}
\tmn

\subsection{Uses}
None

\subsection{Syntax}
\subsubsection{Exported Constants}
None
\subsubsection{Exported Access Programs}

\begin{center}
\begin{tabular}{|l|l|l| p{5cm}|}
\hline
\textbf{Name} & \textbf{In} & \textbf{Out} & \textbf{Exceptions} \\
\hline
new \tmn & & \tmn & \\
\hline
getTreeName & &String & \\
\hline 
setDensity & String & & \\
\hline
getDensity & & String & \\
\hline
setDBH & String & & \\
\hline
getDBH & & String & \\
\hline
setHeight & String & & \\
\hline
getHeight & & String & \\
\hline
setAge & String & & \\
\hline
getAge & & String & \\
\hline
\end{tabular}
\end{center}

\subsection{Semantics}

\subsubsection{State Variables}
$\mathit{Treename: String}$\\
$\mathit{Density: String}$\\
$\mathit{DBH: String}$\\
$\mathit{Height: String}$\\
$\mathit{Age: String}$\\

\subsubsection{Environment Variables}
None

\subsubsection{Assumptions}
None

\subsubsection{Access Routine Semantics}

\noindent new \tmn():
\begin{itemize}
\item transition: $\mathit{Treename, Density, DBH, Height,
Age := ``\constn", ``", ``", ``", ``"}$
\item output: $\mathit{out := self}$
\item exception: None 
\end{itemize}
\noindent getTreeName():
\begin{itemize}
\item transition: None
\item output: $\mathit{out := Treename}$
\item exception: None
\end{itemize}
\noindent setDensity(newDensity):
\begin{itemize}
\item transition: $\mathit{Density := newDensity}$
\item output: None
\item exception: None
\end{itemize}
\noindent getDensity():
\begin{itemize}
\item transition: None
\item output: $\mathit{out := Density}$
\item exception: None
\end{itemize}
\noindent setDBH(newDBH):
\begin{itemize}
\item transition: $\mathit{DBH := newDBH}$
\item output: None
\item exception: None
\end{itemize}
\noindent getDBH():
\begin{itemize}
\item transition: None
\item output: $\mathit{out := DBH}$
\item exception: None
\end{itemize}
\noindent setHeight(newHeight):
\begin{itemize}
\item transition: $\mathit{Height := newHeight}$
\item output: None
\item exception: None
\end{itemize}
\noindent getHeight():
\begin{itemize}
\item transition: None
\item output: $\mathit{out := Height}$
\item exception: None
\end{itemize}
\noindent setAge(newAge):
\begin{itemize}
\item transition: $\mathit{Age := newAge}$
\item output: None
\item exception: None
\end{itemize}
\noindent getAge():
\begin{itemize}
\item transition: None
\item output: $\mathit{out := Age}$
\item exception: None
\end{itemize}

\subsubsection{Local Functions}
None
 %%%%%%%%%%%%%%%%%%%%% Birch ends %%%%%%%%%%%%%

\newpage

%%%%%%%%%%%%%%%%%%% White Pine starts %%%%%%%%%%%
\renewcommand{\tn}{White Pine }
\renewcommand{\tmn}{WhitePine}
\renewcommand{\constn}{White\ Pine}

\section{MIS of \tn (\mref{Model8})}

\subsection{Module}
\tmn

\subsection{Uses}
None

\subsection{Syntax}
\subsubsection{Exported Constants}
None
\subsubsection{Exported Access Programs}

\begin{center}
\begin{tabular}{|l|l|l| p{5cm}|}
\hline
\textbf{Name} & \textbf{In} & \textbf{Out} & \textbf{Exceptions} \\
\hline
new \tmn & & \tmn & \\
\hline
getTreeName & &String & \\
\hline 
setDensity & String & & \\
\hline
getDensity & & String & \\
\hline
setDBH & String & & \\
\hline
getDBH & & String & \\
\hline
setHeight & String & & \\
\hline
getHeight & & String & \\
\hline
setAge & String & & \\
\hline
getAge & & String & \\
\hline
\end{tabular}
\end{center}

\subsection{Semantics}

\subsubsection{State Variables}
$\mathit{Treename: String}$\\
$\mathit{Density: String}$\\
$\mathit{DBH: String}$\\
$\mathit{Height: String}$\\
$\mathit{Age: String}$\\

\subsubsection{Environment Variables}
None

\subsubsection{Assumptions}
None

\subsubsection{Access Routine Semantics}

\noindent new \tmn():
\begin{itemize}
\item transition: $\mathit{Treename, Density, DBH, Height,
Age := ``\constn", ``", ``", ``", ``"}$
\item output: $\mathit{out := self}$
\item exception: None 
\end{itemize}
\noindent getTreeName():
\begin{itemize}
\item transition: None
\item output: $\mathit{out := Treename}$
\item exception: None
\end{itemize}
\noindent setDensity(newDensity):
\begin{itemize}
\item transition: $\mathit{Density := newDensity}$
\item output: None
\item exception: None
\end{itemize}
\noindent getDensity():
\begin{itemize}
\item transition: None
\item output: $\mathit{out := Density}$
\item exception: None
\end{itemize}
\noindent setDBH(newDBH):
\begin{itemize}
\item transition: $\mathit{DBH := newDBH}$
\item output: None
\item exception: None
\end{itemize}
\noindent getDBH():
\begin{itemize}
\item transition: None
\item output: $\mathit{out := DBH}$
\item exception: None
\end{itemize}
\noindent setHeight(newHeight):
\begin{itemize}
\item transition: $\mathit{Height := newHeight}$
\item output: None
\item exception: None
\end{itemize}
\noindent getHeight():
\begin{itemize}
\item transition: None
\item output: $\mathit{out := Height}$
\item exception: None
\end{itemize}
\noindent setAge(newAge):
\begin{itemize}
\item transition: $\mathit{Age := newAge}$
\item output: None
\item exception: None
\end{itemize}
\noindent getAge():
\begin{itemize}
\item transition: None
\item output: $\mathit{out := Age}$
\item exception: None
\end{itemize}

\subsubsection{Local Functions}
None
 %%%%%%%%%%%%%%%%%%%%% White Pine ends %%%%%%%%%%%%%

 \newpage

%%%%%%%%%%%%%%%%%%% Red Maple starts %%%%%%%%%%%
\renewcommand{\tn}{Red Maple }
\renewcommand{\tmn}{RedMaple}
\renewcommand{\constn}{Red\ Maple}

\section{MIS of \tn (\mref{Model9})}

\subsection{Module}
\tmn

\subsection{Uses}
None

\subsection{Syntax}
\subsubsection{Exported Constants}
None
\subsubsection{Exported Access Programs}

\begin{center}
\begin{tabular}{|l|l|l| p{5cm}|}
\hline
\textbf{Name} & \textbf{In} & \textbf{Out} & \textbf{Exceptions} \\
\hline
new \tmn & & \tmn & \\
\hline
getTreeName & &String & \\
\hline 
setDensity & String & & \\
\hline
getDensity & & String & \\
\hline
setDBH & String & & \\
\hline
getDBH & & String & \\
\hline
setHeight & String & & \\
\hline
getHeight & & String & \\
\hline
setAge & String & & \\
\hline
getAge & & String & \\
\hline
\end{tabular}
\end{center}

\subsection{Semantics}

\subsubsection{State Variables}
$\mathit{Treename: String}$\\
$\mathit{Density: String}$\\
$\mathit{DBH: String}$\\
$\mathit{Height: String}$\\
$\mathit{Age: String}$\\

\subsubsection{Environment Variables}
None

\subsubsection{Assumptions}
None

\subsubsection{Access Routine Semantics}

\noindent new \tmn():
\begin{itemize}
\item transition: $\mathit{Treename, Density, DBH, Height,
Age := ``\constn", ``", ``", ``", ``"}$
\item output: $\mathit{out := self}$
\item exception: None 
\end{itemize}
\noindent getTreeName():
\begin{itemize}
\item transition: None
\item output: $\mathit{out := Treename}$
\item exception: None
\end{itemize}
\noindent setDensity(newDensity):
\begin{itemize}
\item transition: $\mathit{Density := newDensity}$
\item output: None
\item exception: None
\end{itemize}
\noindent getDensity():
\begin{itemize}
\item transition: None
\item output: $\mathit{out := Density}$
\item exception: None
\end{itemize}
\noindent setDBH(newDBH):
\begin{itemize}
\item transition: $\mathit{DBH := newDBH}$
\item output: None
\item exception: None
\end{itemize}
\noindent getDBH():
\begin{itemize}
\item transition: None
\item output: $\mathit{out := DBH}$
\item exception: None
\end{itemize}
\noindent setHeight(newHeight):
\begin{itemize}
\item transition: $\mathit{Height := newHeight}$
\item output: None
\item exception: None
\end{itemize}
\noindent getHeight():
\begin{itemize}
\item transition: None
\item output: $\mathit{out := Height}$
\item exception: None
\end{itemize}
\noindent setAge(newAge):
\begin{itemize}
\item transition: $\mathit{Age := newAge}$
\item output: None
\item exception: None
\end{itemize}
\noindent getAge():
\begin{itemize}
\item transition: None
\item output: $\mathit{out := Age}$
\item exception: None
\end{itemize}

\subsubsection{Local Functions}
None
 %%%%%%%%%%%%%%%%%%%%% Red Maple ends %%%%%%%%%%%%%

\newpage

%%%%%%%%%%%%%%%%%%% Red Oak starts %%%%%%%%%%%
\renewcommand{\tn}{Red Oak }
\renewcommand{\tmn}{RedOak}
\renewcommand{\constn}{Red\ Oak}

\section{MIS of \tn (\mref{Model10})}

\subsection{Module}
\tmn

\subsection{Uses}
None

\subsection{Syntax}
\subsubsection{Exported Constants}
None
\subsubsection{Exported Access Programs}

\begin{center}
\begin{tabular}{|l|l|l| p{5cm}|}
\hline
\textbf{Name} & \textbf{In} & \textbf{Out} & \textbf{Exceptions} \\
\hline
new \tmn & & \tmn & \\
\hline
getTreeName & &String & \\
\hline 
setDensity & String & & \\
\hline
getDensity & & String & \\
\hline
setDBH & String & & \\
\hline
getDBH & & String & \\
\hline
setHeight & String & & \\
\hline
getHeight & & String & \\
\hline
setAge & String & & \\
\hline
getAge & & String & \\
\hline
\end{tabular}
\end{center}

\subsection{Semantics}

\subsubsection{State Variables}
$\mathit{Treename: String}$\\
$\mathit{Density: String}$\\
$\mathit{DBH: String}$\\
$\mathit{Height: String}$\\
$\mathit{Age: String}$\\

\subsubsection{Environment Variables}
None

\subsubsection{Assumptions}
None

\subsubsection{Access Routine Semantics}

\noindent new \tmn():
\begin{itemize}
\item transition: $\mathit{Treename, Density, DBH, Height,
Age := ``\constn", ``", ``", ``", ``"}$
\item output: $\mathit{out := self}$
\item exception: None 
\end{itemize}
\noindent getTreeName():
\begin{itemize}
\item transition: None
\item output: $\mathit{out := Treename}$
\item exception: None
\end{itemize}
\noindent setDensity(newDensity):
\begin{itemize}
\item transition: $\mathit{Density := newDensity}$
\item output: None
\item exception: None
\end{itemize}
\noindent getDensity():
\begin{itemize}
\item transition: None
\item output: $\mathit{out := Density}$
\item exception: None
\end{itemize}
\noindent setDBH(newDBH):
\begin{itemize}
\item transition: $\mathit{DBH := newDBH}$
\item output: None
\item exception: None
\end{itemize}
\noindent getDBH():
\begin{itemize}
\item transition: None
\item output: $\mathit{out := DBH}$
\item exception: None
\end{itemize}
\noindent setHeight(newHeight):
\begin{itemize}
\item transition: $\mathit{Height := newHeight}$
\item output: None
\item exception: None
\end{itemize}
\noindent getHeight():
\begin{itemize}
\item transition: None
\item output: $\mathit{out := Height}$
\item exception: None
\end{itemize}
\noindent setAge(newAge):
\begin{itemize}
\item transition: $\mathit{Age := newAge}$
\item output: None
\item exception: None
\end{itemize}
\noindent getAge():
\begin{itemize}
\item transition: None
\item output: $\mathit{out := Age}$
\item exception: None
\end{itemize}

\subsubsection{Local Functions}
None
 %%%%%%%%%%%%%%%%%%%%% Red Oak ends %%%%%%%%%%%%%

\newpage

%%%%%%%%%%%%%%%%%%% Environmental Data starts %%%%%%%%%%%
\noindent
\textcolor{red}{We deleted ``isValidString'' local 
function here because we will 
check the validity of string when users enter data from the GUI}
\section{MIS of Environmental Data (\mref{Model11})}

\subsection{Module}
EnvData

\subsection{Uses}
None

\subsection{Syntax}
\subsubsection{Exported Constants}
None
\subsubsection{Exported Access Programs}

\begin{center}
\begin{tabular}{|l|l|l| p{5cm}|}
\hline
\textbf{Name} & \textbf{In} & \textbf{Out} & \textbf{Exceptions} \\
\hline
new EnvData & & EnvData & \\
\hline

setHumidity & String & & \\
\hline
getHumidity & &String & \\
\hline

setTemp & String & &\\
\hline
getTemp & &String & \\
\hline

setSC & String & & \\
\hline
getSC & &String & \\
\hline

setSN & String & & \\
\hline
getSN & &String & \\
\hline

setLAI & String & & \\
\hline
getLAI & &String & \\
\hline
\end{tabular}
\end{center}

\subsection{Semantics}

\subsubsection{State Variables}
$\mathit{Humility: String}$\\
$\mathit{Temp: String}$\\
$\mathit{SC: String}$\\
$\mathit{SN: String}$\\
$\mathit{LAI: String}$\\

\subsubsection{Environment Variables}
None

\subsubsection{Assumptions}
None

\subsubsection{Access Routine Semantics}

\noindent new EnvData():
\begin{itemize}
\item transition: $\mathit{Humidity, Temp, SC, SN, LAI := ``", ``", ``", ``", ``"}$
\item output: $\mathit{out := self}$
\item exception: None 
\end{itemize}

\newcommand{\attr}{Humidity}
\noindent get\attr():
\begin{itemize}
\item transition: None
\item output: $\mathit{out := \attr}$
\item exception: None
\end{itemize}

\noindent set\attr(new\attr):
\begin{itemize}
\item transition: $\mathit{\attr := new\attr}$
\item output: None
\item exception: None
\end{itemize}

\renewcommand{\attr}{Temp}
\noindent get\attr():
\begin{itemize}
\item transition: None
\item output: $\mathit{out := \attr}$
\item exception: None
\end{itemize}

\noindent set\attr(new\attr):
\begin{itemize}
\item transition: $\mathit{\attr := new\attr}$
\item output: None
\item exception: None
\end{itemize}

\renewcommand{\attr}{SC}
\noindent get\attr():
\begin{itemize}
\item transition: None
\item output: $\mathit{out := \attr}$
\item exception: None
\end{itemize}

\noindent set\attr(new\attr):
\begin{itemize}
\item transition: $\mathit{\attr := new\attr}$
\item output: None
\item exception: None
\end{itemize}

\renewcommand{\attr}{SN}
\noindent get\attr():
\begin{itemize}
\item transition: None
\item output: $\mathit{out := \attr}$
\item exception: None
\end{itemize}

\noindent set\attr(new\attr):
\begin{itemize}
\item transition: $\mathit{\attr := new\attr}$
\item output: None
\item exception: None
\end{itemize}

\renewcommand{\attr}{LAI}
\noindent get\attr():
\begin{itemize}
\item transition: None
\item output: $\mathit{out := \attr}$
\item exception: None
\end{itemize}

\noindent set\attr(new\attr):
\begin{itemize}
\item transition: $\mathit{\attr := new\attr}$
\item output: None
\item exception: None
\end{itemize}

\subsubsection{Local Functions}
None
 %%%%%%%%%%%%%%%%%%%%% Environmental data ends %%%%%%%%%%%%%

\newpage

%%%%%%%%%%%%%%%% Plot Data starts %%%%%%%%%%%%%%%
\section{MIS of Plot Data (\mref{Model12})}

\subsection{Module}
PlotData

\subsection{Uses}
\mref{Model4}, \mref{Model5}, \mref{Model6}, \mref{Model7},
\mref{Model8}, \mref{Model9}, \mref{Model10}, \mref{Model11}    

\subsection{Syntax}
\subsubsection{Exported Constants}
None
\subsubsection{Exported Access Programs}

\begin{center}
\begin{tabular}{|l|l|l| p{5cm}|}
\hline
\textbf{Name} & \textbf{In} & \textbf{Out} & \textbf{Exceptions} \\
\hline
new PlotData & & PlotData & \\
\hline

setRedPineObj & RedPine  & & \\
\hline
getRedPineObj & & RedPine & \\
\hline

setOakObj & Oak  & & \\
\hline
getOakObj & & Oak & \\
\hline

setBeechObj & Beech  & & \\
\hline
getBeechObj & & Beech & \\
\hline

setBirchObj & Birch  & & \\
\hline
getBirchObj & & Birch & \\
\hline

setWhitePineObj & WhitePine  & & \\
\hline
getWhitePineObj & & WhitePine & \\
\hline

setRedMapleObj & RedMaple  & & \\
\hline
getRedMapleObj & & RedMaple & \\
\hline

setRedOakObj & RedOak  & & \\
\hline
getRedOakObj & & RedOak & \\
\hline

setEnvDataObj & EnvData  & & \\
\hline
getEnvDataObj & & EnvData & \\
\hline

\end{tabular}
\end{center}

\newpage

\subsection{Semantics}

\subsubsection{State Variables}
$\mathit{RedPineObj: RedPine}$\\
$\mathit{OakObj: Oak}$\\
$\mathit{BeechObj: Beech}$\\
\renewcommand{\attr}{Birch}
$\mathit{\attr Obj: \attr}$\\
\renewcommand{\attr}{WhitePine}
$\mathit{\attr Obj: \attr}$\\
\renewcommand{\attr}{RedMaple}
$\mathit{\attr Obj: \attr}$\\
\renewcommand{\attr}{RedOak}
$\mathit{\attr Obj: \attr}$\\
\renewcommand{\attr}{EnvData}
$\mathit{\attr Obj: \attr}$\\

\subsubsection{Environment Variables}
None

\subsubsection{Assumptions}
None

\subsubsection{Access Routine Semantics}

\noindent new PlotData():
\begin{itemize}
\item transition: 
\begin{itemize}
\item $\mathit{RedPineObj, OakObj, BeechObj, 
BirchObj := null, null, null, null}$

\item $\mathit{WhitePineObj, RedMapleObj, RedOakObj, EvnDataObj
:= null, null, null, null}$
\end{itemize}

\item output: $\mathit{out := self}$
\item exception: None 
\end{itemize}


\renewcommand{\attr}{RedPineObj}
\noindent get\attr():
\begin{itemize}
\item transition: None
\item output: $\mathit{out := \attr}$
\item exception: None
\end{itemize}
\noindent set\attr(new\attr):
\begin{itemize}
\item transition: $\mathit{\attr := new\attr}$
\item output: None
\item exception: None
\end{itemize}
\renewcommand{\attr}{OakObj}
\noindent get\attr():
\begin{itemize}
\item transition: None
\item output: $\mathit{out := \attr}$
\item exception: None
\end{itemize}
\noindent set\attr(new\attr):
\begin{itemize}
\item transition: $\mathit{\attr := new\attr}$
\item output: None
\item exception: None
\end{itemize}
\renewcommand{\attr}{BeechObj}
\noindent get\attr():
\begin{itemize}
\item transition: None
\item output: $\mathit{out := \attr}$
\item exception: None
\end{itemize}
\noindent set\attr(new\attr):
\begin{itemize}
\item transition: $\mathit{\attr := new\attr}$
\item output: None
\item exception: None
\end{itemize}
\renewcommand{\attr}{BirchObj}
\noindent get\attr():
\begin{itemize}
\item transition: None
\item output: $\mathit{out := \attr}$
\item exception: None
\end{itemize}
\noindent set\attr(new\attr):
\begin{itemize}
\item transition: $\mathit{\attr := new\attr}$
\item output: None
\item exception: None
\end{itemize}
\renewcommand{\attr}{WhitePineObj}
\noindent get\attr():
\begin{itemize}
\item transition: None
\item output: $\mathit{out := \attr}$
\item exception: None
\end{itemize}
\noindent set\attr(new\attr):
\begin{itemize}
\item transition: $\mathit{\attr := new\attr}$
\item output: None
\item exception: None
\end{itemize}
\renewcommand{\attr}{RedMapleObj}
\noindent get\attr():
\begin{itemize}
\item transition: None
\item output: $\mathit{out := \attr}$
\item exception: None
\end{itemize}
\noindent set\attr(new\attr):
\begin{itemize}
\item transition: $\mathit{\attr := new\attr}$
\item output: None
\item exception: None
\end{itemize}
\renewcommand{\attr}{RedOakObj}
\noindent get\attr():
\begin{itemize}
\item transition: None
\item output: $\mathit{out := \attr}$
\item exception: None
\end{itemize}
\noindent set\attr(new\attr):
\begin{itemize}
\item transition: $\mathit{\attr := new\attr}$
\item output: None
\item exception: None
\end{itemize}
\renewcommand{\attr}{EnvDataObj}
\noindent get\attr():
\begin{itemize}
\item transition: None
\item output: $\mathit{out := \attr}$
\item exception: None
\end{itemize}
\noindent set\attr(new\attr):
\begin{itemize}
\item transition: $\mathit{\attr := new\attr}$
\item output: None
\item exception: None
\end{itemize}


\subsubsection{Local Functions}
None
%%%%%%%%%%%%%%%% Plot Data Ends %%%%%%%%%%%%%%%%%

\newpage

%%%%%%%%%%%%%%%%%  First Person Player starts %%%%%%%
\section{MIS of First Person Player (\mref{Model13})}

\subsection{Module}
FirstPersonPlayer

\subsection{Uses}
Character Controller Module from Unity

\subsection{Syntax}
This is a module provided by UnityEngine.UI. Please click
\href{https://docs.unity3d.com/ScriptReference/CharacterController.html}{here} to check offical document from Unity. We have 
designed a controller for this module. The controller is 
PlayerMovement(\mref{Controller4}).
\subsection{Semantics}
This is a module provided by UnityEngine.UI. Please click
\href{https://docs.unity3d.com/ScriptReference/CharacterController.html}{here} to check offical document from Unity. We have 
designed a controller for this module. The controller is 
PlayerMovement(\mref{Controller4}).

%%%%%%%%%%%%%%%%%% First Person Player ends %%%%%%%%%

\newpage

%%%%%%%%%%%%%%%%%%%%%%%%%%%% Json File Start %%%%%%%%%
\section{MIS of Json File (\mref{Model14})}

\subsection{Module}
JsonFile. This is not a typical class. This section only aims
to show how JSON files are organized formally.

\subsection{Local Type}
$X = tuple(key:String,\ value:String)$ $\land$ 
isValidString(value) \\
$S : set\ of\ X$\\
$TreeANDEnvData = tuple(key:String,\ values: S)$

\subsection{State Variables}
$JsonFile : set\ of\ TreeANDEnvData$

\newcounter{var}

\subsection{Example}
\begin{itemize}
    \item 
First, define all the tuples that have type $X$.
\begin{itemize}
    \item $x_1 = ("DBH", "10")\ :X$
    \item $x_2 = ("Age", "10")\ :X$
    \item $x_3 = ("Height", "10")\ :X$
    \item $x_4 = ("Density", "10")\ :X$
    \vspace{0.5cm}

    \item $x_5 = ("DBH", "20")\ :X$
    \item $x_6 = ("Age", "20")\ :X$
    \item $x_7 = ("Height", "20")\ :X$
    \item $x_8 = ("Density", "20")\ :X$
    \vspace{0.5cm}
    
    \item $x_9 = ("DBH", "30")\ :X$
    \item $x_{10} = ("Age", "30")\ :X$
    \item $x_{11} = ("Height", "30")\ :X$
    \item $x_{12} = ("Density", "30")\ :X$
    \vspace{0.5cm}
    
    \renewcommand{\attr}{40}
    \item $x_{13} = ("DBH", "\attr")\ :X$
    \item $x_{14} = ("Age", "\attr")\ :X$
    \item $x_{15} = ("Height", "\attr")\ :X$
    \item $x_{16} = ("Density", "\attr")\ :X$
    \vspace{0.5cm}

    \renewcommand{\attr}{50}
    \item $x_{17} = ("DBH", "\attr")\ :X$
    \item $x_{18} = ("Age", "\attr")\ :X$
    \item $x_{19} = ("Height", "\attr")\ :X$
    \item $x_{20} = ("Density", "\attr")\ :X$
    \vspace{0.5cm}

    \renewcommand{\attr}{60}
    \item $x_{21} = ("DBH", "\attr")\ :X$
    \item $x_{22} = ("Age", "\attr")\ :X$
    \item $x_{23} = ("Height", "\attr")\ :X$
    \item $x_{24} = ("Density", "\attr")\ :X$
    \vspace{0.5cm}

    \renewcommand{\attr}{70}
    \item $x_{25} = ("DBH", "\attr")\ :X$
    \item $x_{26} = ("Age", "\attr")\ :X$
    \item $x_{27} = ("Height", "\attr")\ :X$
    \item $x_{28} = ("Density", "\attr")\ :X$
    \vspace{0.5cm}

    
    \item $x_{29} = ("Humility", "10")\ :X$
    \item $x_{30} = ("Temperature", "20")\ :X$
    \item $x_{31} = ("SC", "10")\ :X$
    \item $x_{32} = ("SN", "95")\ :X$
    \item $x_{33} = ("LAI", "95")\ :X$
\end{itemize}

    \item
Second, define all the sets that have type $S$
\begin{itemize}
    \item $s_1 = \{x_1, x_2, x_3, x_4\}\ :S$
    \item $s_2 = \{x_5, x_6, x_7, x_8\}\ :S$
    \item $s_3 = \{x_9, x_{10}, x_{11}, x_{12}\}\ :S$
    \item $s_4 = \{x_{13}, x_{14}, x_{15}, x_{16}\}\ :S$
    \item $s_5 = \{x_{17}, x_{18}, x_{19}, x_{20}\}\ :S$
    \item $s_6 = \{x_{21}, x_{22}, x_{23}, x_{24}\}\ :S$
    \item $s_7 = \{x_{25}, x_{26}, x_{27}, x_{28}\}\ :S$
    \item $s_8 = \{x_{29}, x_{30}, x_{31}, x_{32}, x_{33}\}\ :S$
\end{itemize}
    \newpage
    \item
Third, define all the tuples that have type $TreeANDEnvData$.
\begin{itemize}
    \item $d_1 = ("RedPineData", s_1):\ TreeANDEnvData$
    \item $d_2 = ("OakData", s_2):\ TreeANDEnvData$
    \item $d_3 = ("BeechData", s_3):\ TreeANDEnvData$
    \item $d_4 = ("BirchData", s_4):\ TreeANDEnvData$
    \item $d_5 = ("WhitePineData", s_5):\ TreeANDEnvData$
    \item $d_6 = ("RedMapleData", s_6):\ TreeANDEnvData$
    \item $d_7 = ("RedOakData", s_7):\ TreeANDEnvData$
    \item $d_8 = ("EnvData", s_8):\ TreeANDEnvData$
\end{itemize}
    \item
    Finally, $JsonFile = \{d_1, d_2, d_3, d_4, d_5, d_6,
    d_7, d_8\}.$
\end{itemize}

\subsection{Local Functions}
ValidCharacters = \{``1", ``2", ``3", ``4", ``5", ``6", ``7"
, ``8", ``9", ``0", ``."\}\\

\noindent isValidString(S): String $\rightarrow$ $\mathbb{B}$ \\

\noindent isValidString(S) = $\forall(i : \mathbb{Z} | 0 \leq
i < |S| : S[i] \in $ ValidCharacters)
%%%%%%%%%%%%%%%%%%%%%%%%%%%% Json File Ends %%%%%%%%%%

%%%%%%%%%%%%%% TS Ends %%%%%%%%%%%%%%%%%%%%%%%%%%%%%%%%%%%%%%%%

\newpage

%%%%%%%%%%%%%%%%% MainPage starts %%%%%%%%%%%%%%
\section{MIS of Main Page (\mref{Viwer1})}

\subsection{Module}
MainPageDisplay

\subsection{Uses}
\mref{Controller6} , \mref{Controller7}, \mref{Controller8}, \mref{Controller9}, \mref{Controller10}, \mref{Viwer2}, \mref{Viwer3}, \mref{Viwer4}, \mref{Viwer5}
UnityEngine.UI

\subsection{Syntax}
\subsubsection{Exported Constants}
None
\subsubsection{Exported Access Programs}
None

\subsection{Semantics}
This module is used to display the UI of the homepage.  You can refer to Unity Canvas Documentation by clicking \href{https://docs.unity3d.com/Packages/com.unity.ugui@1.0/manual/class-Canvas.html}{here}.
\subsubsection{State Variables}
None
\subsubsection{Environment Variables}
None
\subsubsection{Assumptions}
None
\subsubsection{Access Routine Semantics}
None
\subsubsection{Local Functions}
None
%%%%%%%%%%%%%%%% Mainpage ends %%%%%%%%%%%%%


\newpage


\renewcommand{\bref}{\href{https://docs.unity3d.com/Packages/com.unity.ugui@1.0/manual/script-Button.html}{here}}

%%%%%%%%%%%%%%%%% StartButton starts %%%%%%%%%%%%%%
\section{MIS of Start Button (\mref{Viwer2})}

\subsection{Module}
StartButton

\subsection{Uses}
\mref{Controller6} , 
UnityEngine.UI

\subsection{Syntax}
\subsubsection{Exported Constants}
None
\subsubsection{Exported Access Programs}
None

\subsection{Semantics}
This module is used to display the UI of the StartButton. You can refer to Unity Button Documentation by clicking \bref.
\subsubsection{State Variables}
None
\subsubsection{Environment Variables}
windows: Computer screen used to display messages.
\subsubsection{Assumptions}
None
\subsubsection{Access Routine Semantics}
None
\subsubsection{Local Functions}
None
%%%%%%%%%%%%%%%% StartButton ends %%%%%%%%%%%%%


\newpage

%%%%%%%%%%%%%% Instruction button starts %%%%%%%%%%%%%
\section{MIS of Instruction Button (\mref{Viwer3})}

\subsection{Module}
InstructionButton

\subsection{Uses}
\mref{Controller7}  , UnityEngine.UI

\subsection{Syntax}
\subsubsection{Exported Constants}
None
\subsubsection{Exported Access Programs}
None

\subsection{Semantics}
This module is used to display the UI of the InstructionButton.You can refer to Unity Button Documentation by clicking \bref.
\subsubsection{State Variables}
None
\subsubsection{Environment Variables}
windows: Computer screen used to display messages.
\subsubsection{Assumptions}
None
\subsubsection{Access Routine Semantics}
None
\subsubsection{Local Functions}
None
%%%%%%%%%%%% InstructionButton ends %%%%%%%%%%%%%

\newpage

%%%%%%%%%%%%%%%% Contact Us Button %%%%%%%%%%%%%%%%
\section{MIS of Contact Us Button (\mref{Viwer4})}

\subsection{Module}
ContactUsButton

\subsection{Uses}
\mref{Controller8}  , UnityEngine.UI

\subsection{Syntax}
\subsubsection{Exported Constants}
None
\subsubsection{Exported Access Programs}
None

\subsection{Semantics}
This module is used to display the UI of the ContactUsButton.You can refer to Unity Button Documentation by clicking \bref.
\subsubsection{State Variables}
None
\subsubsection{Environment Variables}
windows: Computer screen used to display messages.
\subsubsection{Assumptions}
None
\subsubsection{Access Routine Semantics}
None
\subsubsection{Local Functions}
None
%%%%%%%%%%%%% contact us button ends %%%%%%%%%%%%%%%%%

\newpage

%%%%%%%%%%%%%%%%%%% Quit button starts %%%%%%%%%%%%%%%
\section{MIS of Quit Button (\mref{Viwer5})}

\subsection{Module}
QuitButton

\subsection{Uses}
\mref{Controller10} ,UnityEngine.UI 

\subsection{Syntax}
\subsubsection{Exported Constants}
None
\subsubsection{Exported Access Programs}
None

\subsection{Semantics}
This module is used to display the UI of the QuitButton.You can refer to Unity Button Documentation by clicking \bref.
\subsubsection{State Variables}
None
\subsubsection{Environment Variables}
windows: Computer screen used to display messages.
\subsubsection{Assumptions}
None
\subsubsection{Access Routine Semantics}
None
\subsubsection{Local Functions}
None
%%%%%%%%%%%%%%%% Quit button ends %%%%%%%%%%%%%%%%%%%

\newpage

%%%%%%%%%%%%%%%%% InstructionInfoDisplay starts %%%%%%%%%%%%%%
\section{MIS of Instruction Page (\mref{Viwer6})}

\subsection{Module}
InstructionInfoDisplay

\subsection{Uses}
\mref{Controller7} ,
UnityEngine.UI

\subsection{Syntax}
\subsubsection{Exported Constants}
None
\subsubsection{Exported Access Programs}
None

\subsection{Semantics}
This module is used to display the UI of the instruction page. 
You can refer to Unity Canvas Documentation by clicking 
\href{https://docs.unity3d.com/Packages/com.unity.ugui@1.0/manual/class-Canvas.html}{here}.
\subsubsection{State Variables}
None
\subsubsection{Environment Variables}
windows: Computer screen used to display messages.
\subsubsection{Assumptions}
None
\subsubsection{Access Routine Semantics}
None
\subsubsection{Local Functions}
None
%%%%%%%%%%%%%%%% InstructionInfoDisplay ends %%%%%%%%%%%%%

\newpage

%%%%%%%%%%%%%%%%% ContactUsInfoDisplay starts %%%%%%%%%%%%%%
\section{MIS of Contact Us Page (\mref{Viwer7})}

\subsection{Module}
ContactUsInfoDisplay

\subsection{Uses}
\mref{Controller9} 
UnityEngine.UI

\subsection{Syntax}
\subsubsection{Exported Constants}
None
\subsubsection{Exported Access Programs}
None

\subsection{Semantics}
This module is used to display the UI of the Contact Us page.
You can refer to Unity Canvas Documentation by clicking 
\href{https://docs.unity3d.com/Packages/com.unity.ugui@1.0/manual/class-Canvas.html}{here}.

\subsubsection{State Variables}
None
\subsubsection{Environment Variables}
windows: Computer screen used to display messages.
\subsubsection{Assumptions}
None
\subsubsection{Access Routine Semantics}
None
\subsubsection{Local Functions}
None
%%%%%%%%%%%%%%%% ContactUsInfoDisplay ends %%%%%%%%%%%%%


\newpage

%%%%%%%%%%%%%%%%% Back button starts %%%%%%%%%%%
\section{MIS of Back Button (\mref{Viwer8})}

\subsection{Module}
BackButton

\subsection{Uses}
\mref{Controller11}  , UnityEngine.UI

\subsection{Syntax}
\subsubsection{Exported Constants}
None
\subsubsection{Exported Access Programs}
None

\subsection{Semantics}
This module is used to display the UI of the BackButton.You can refer to Unity Button Documentation by clicking \bref.
\subsubsection{State Variables}
None
\subsubsection{Environment Variables}
windows: Computer screen used to display messages.
\subsubsection{Assumptions}
None
\subsubsection{Access Routine Semantics}
None
\subsubsection{Local Functions}
None
%%%%%%%%%%%%%% Back button ends %%%%%%%%%%%%%%%

\newpage

%%%%%%%%%%%%%%%%% UpdateDataDisplay starts %%%%%%%%%%%%%%
\section{MIS of Update Data Page (\mref{Viwer9})}

\subsection{Module}
UpdateDataDisplay

\subsection{Uses}
\mref{Controller8} 
UnityEngine.UI

\subsection{Syntax}
\subsubsection{Exported Constants}
None
\subsubsection{Exported Access Programs}
None

\subsection{Semantics}
This module is used to display the UI of the Update Data page.
You can refer to Unity Canvas Documentation by clicking 
\href{https://docs.unity3d.com/Packages/com.unity.ugui@1.0/manual/class-Canvas.html}{here}.
\subsubsection{State Variables}
None
\subsubsection{Environment Variables}
windows: Computer screen used to display messages.
\subsubsection{Assumptions}
None
\subsubsection{Access Routine Semantics}
None
\subsubsection{Local Functions}
None
%%%%%%%%%%%%%%%% UpdateDataDisplay ends %%%%%%%%%%%%%


\newpage

%%%%%%%%%%%%%%%%%%%%%%%%%%%%%%%%%%%%%%%%%%%%%%%%
\section{MIS of Environmental Data Selection Button
 (\mref{Viwer10})}

\subsection{Module}
EnvDataSelectionButton

\subsection{Uses}
\mref{Controller16}  , UnityEngine.UI

\subsection{Syntax}
\subsubsection{Exported Constants}
None
\subsubsection{Exported Access Programs}
None

\subsection{Semantics}
This module is used to display the UI of the EnvDataSelectionButton.You can refer to Unity Button Documentation by clicking \bref.
\subsubsection{State Variables}
None
\subsubsection{Environment Variables}
windows: Computer screen used to display messages.
\subsubsection{Assumptions}
None
\subsubsection{Access Routine Semantics}
None
\subsubsection{Local Functions}
None
%%%%%%%%%%%%%%%%%%%%%%%%%%%%%%%%%%%%%%%%%%%%%%%

\newpage

%%%%%%%%%%%%%%%%%%%%%%%%%%%%%%%%%%%%%%%%%%%%%%%
\section{MIS of Data Type Selection Button (\mref{Viwer11})}

\subsection{Module}
DataTypeSelectionButton

\subsection{Uses}
\mref{Controller17}  , UnityEngine.UI

\subsection{Syntax}
\subsubsection{Exported Constants}
None
\subsubsection{Exported Access Programs}
None

\subsection{Semantics}
This module is used to display the UI of the DataTypeSelectionButton.You can refer to Unity Button Documentation by clicking \bref.
\subsubsection{State Variables}
None
\subsubsection{Environment Variables}
windows: Computer screen used to display messages.
\subsubsection{Assumptions}
None
\subsubsection{Access Routine Semantics}
None
\subsubsection{Local Functions}
None
%%%%%%%%%%%%%%%%%%%%%%%%%%%%%%%%%%%%%%%%%%%%%%%%%%

\newpage

%%%%%%%%%%%%%%%%% NewDataInputBox starts %%%%%%%%%%%%%%
\section{MIS of New Data Input Box (\mref{Viwer12})}

\subsection{Module}
NewDataInputBox

\subsection{Uses} ,
UnityEngine.UI

\subsection{Syntax}
\subsubsection{Exported Constants}
None
\subsubsection{Exported Access Programs}
None

\subsection{Semantics}
This module is used to display the UI of the new data input box.
You can refer to Unity Input Field Documentation by clicking 
\href{https://docs.unity3d.com/Packages/com.unity.ugui@1.0/manual/script-InputField.html}{here}
\subsubsection{State Variables}
None
\subsubsection{Environment Variables}
windows: Computer screen used to display messages.
\subsubsection{Assumptions}
None
\subsubsection{Access Routine Semantics}
None
\subsubsection{Local Functions}
None
%%%%%%%%%%%%%%%% NewDataInputBox ends %%%%%%%%%%%%%


\newpage

%%%%%%%%%%%%%%%%%%%%%%%%%%%%%%%%%%%%%%%%%%%%%%%%%%
\section{MIS of Save Button (\mref{Viwer13})}

\subsection{Module}
SaveButton

\subsection{Uses}
\mref{Controller18}  , UnityEngine.UI

\subsection{Syntax}
\subsubsection{Exported Constants}
None
\subsubsection{Exported Access Programs}
None

\subsection{Semantics}
This module is used to display the UI of the SaveButton.You can refer to Unity Button Documentation by clicking \bref.
\subsubsection{State Variables}
None
\subsubsection{Environment Variables}
windows: Computer screen used to display messages.
\subsubsection{Assumptions}
None
\subsubsection{Access Routine Semantics}
None
\subsubsection{Local Functions}
None
%%%%%%%%%%%%%%%%%%%%%%%%%%%%%%%%%%%%%%%%%%%%%%%%%%%

\newpage


\newcommand{\tref}{\href{https://docs.unity3d.com/Packages/com.unity.ugui@1.0/manual/script-Text.html}{here}}

%%%%%%%%%%% CurrentDataDisplay starts %%%%%%%%%%%%%%%%%%
\section{MIS of Current Data Dispaly (\mref{Viwer14})} 

\subsection{Module}
CurrentDataDisplay

\subsection{Uses}
UnityEngine.UI 

\subsection{Syntax}
\subsubsection{Exported Constants}
None
\subsubsection{Exported Access Programs}
None

\subsection{Semantics}
This module is used to display the UI of the current data. 
You can refer to Unity Text Documentation by clicking \tref.
\subsubsection{State Variables}
None
\subsubsection{Environment Variables}
windows: Computer screen used to display messages.
\subsubsection{Assumptions}
None
\subsubsection{Access Routine Semantics}
None
\subsubsection{Local Functions}
None
%%%%%%%%%%%%%%%%%%%%%%%%% CurrentDataDisplay ends %%%%%%%%%%%%%

\newpage

\newcommand{\dref}{\href{https://docs.unity3d.com/Packages/com.unity.ugui@1.0/manual/script-Dropdown.html}{here}}
%%%%%%%%%%% PlotSelectionDropDown starts %%%%%%%%%%%%%%%%%%
\section{MIS of Plot Selection Drop Down (\mref{Viwer15})} 

\subsection{Module}

PlotSelection

\subsection{Uses}

\mref{Controller12}, UnityEngine.UI

\subsection{Syntax}

\subsubsection{Exported Constants}
None
\subsubsection{Exported Access Programs}
None

\subsection{Semantics}
This module is used to display the dropdown box of plot 
selection. You can refer to Unity Drop Down Documentation 
by clicking \dref.
\subsubsection{State Variables}
None
\subsubsection{Environment Variables}
windows: Computer screen used to display messages.
\subsubsection{Assumptions}
None
\subsubsection{Access Routine Semantics}
None
\subsubsection{Local Functions}
None
%%%%%%%%%%%%%%%%%%%%%%%%% PlotSelectionDropDown ends %%%%%%%%%

\newpage

%%%%%%%%%%% TreeTypeSelectionDropDown starts %%%%%%%%%%%%%%%%%%
\section{MIS of Tree Type Selection Drop Down (\mref{Viwer16})} 

\subsection{Module}

TreeTypeSelection

\subsection{Uses}

\mref{Controller13} , UnityEngine.UI

\subsection{Syntax}

\subsubsection{Exported Constants}
None
\subsubsection{Exported Access Programs}
None

\subsection{Semantics}
This module is used to display the dropdown box of the tree type
selection. You can refer to Unity Drop Down Documentation 
by clicking \dref.

\subsubsection{State Variables}
None
\subsubsection{Environment Variables}
windows: Computer screen used to display messages.
\subsubsection{Assumptions}
None
\subsubsection{Access Routine Semantics}
None
\subsubsection{Local Functions}
None

%%%%%%%%%%%%%%%%%%%%%%%%% TreeTypeSelectionDropDown ends %%%%%%

\newpage

%%%%%%%%%%%%%%%%%%%%%%%%%%%%%%%%%%%%%%%%%%%
\section{MIS of Update Data Button (\mref{Viwer17})}

\subsection{Module}
UpdateDataButton

\subsection{Uses}
\mref{Controller8} , UnityEngine.UI 

\subsection{Syntax}
\subsubsection{Exported Constants}
None
\subsubsection{Exported Access Programs}
None

\subsection{Semantics}
The module is used to display the UI of UpdateDataButton. You can refer to Unity Button Documentation by clicking \bref.
\subsubsection{State Variables}
None
\subsubsection{Environment Variables}
windows: Computer screen used to display messages.
\subsubsection{Assumptions}
None
\subsubsection{Access Routine Semantics}
None
\subsubsection{Local Functions}
None
%%%%%%%%%%%%%%%%%%%%%%%%%%%%%%%%%%%%%%%%%%%%%%%%%%%%

\newpage

%%%%%%%%%%% ForestDisplay starts %%%%%%%%%%%%%%%%%%
\section{MIS of Forest Dispaly (\mref{Viwer18})} 

\subsection{Module}
ForestDisplay

\subsection{Uses}
UnityEngine.UI, \mref{Model1}, \mref{Model2}, \mref{Model3}

\subsection{Syntax}

\subsubsection{Exported Constants}
None
\subsubsection{Exported Access Programs}
None

\subsection{Semantics}
\subsubsection{State Variables}
This module is used to display the forest models.
\subsubsection{Environment Variables}
windows: Computer screen used to display messages.
\subsubsection{Assumptions}
None
\subsubsection{Access Routine Semantics}
None
\subsubsection{Local Functions}
None
%%%%%%%%%%%%%%%%%%%%%%%%% ForestDisplay ends %%%%%%%%%%%%%

\newpage

%%%%%%%%%%%%%%%%%%%%%%%%%%%%%%%%%%%%%%%%%%%%%%%%%%%%%
\section{MIS of Show Environmental Data Button (\mref{Viwer19})}

\subsection{Module}
ShowEnvDataButton

\subsection{Uses}
\mref{Controller14}  , UnityEngine.UI

\subsection{Syntax}
\subsubsection{Exported Constants}
None
\subsubsection{Exported Access Programs}
None


\subsection{Semantics}
This module is used to display the UI of the 
ShowEnvDataButton.You can refer to Unity Button Documentation by
clicking \bref.
\subsubsection{State Variables}
None
\subsubsection{Environment Variables}
windows: Computer screen used to display messages.
\subsubsection{Assumptions}
None
\subsubsection{Access Routine Semantics}
None
\subsubsection{Local Functions}
None

%%%%%%%%%%%%%%%%%%%%%%%%%%%%%%%%%%%%%%%%%%%%%%%%

\newpage

%%%%%%%%%%%%%%%%%%%%%%%%%%%%%%%%%%%%%%%%%%%%%%%%%
\section{MIS of Show Tree Parameters Button (\mref{Viwer20})}

\subsection{Module}
ShowTreeParamButton

\subsection{Uses}
\mref{Controller15}  , UnityEngine.UI

\subsection{Syntax}
\subsubsection{Exported Constants}
None
\subsubsection{Exported Access Programs}
None


\subsection{Semantics}
This module is used to display the UI of the 
ShowTreeParamButton.You can refer to Unity Button Documentation by
clicking \bref.

\subsubsection{State Variables}
None

\subsubsection{Environment Variables}
windows: Computer screen used to display messages.

\subsubsection{Assumptions}
None

\subsubsection{Access Routine Semantics}
None



\subsubsection{Local Functions}
None

\newpage


%%%%%%%%%%% EnvDataDisplay starts %%%%%%%%%%%%%%%%%%
\section{MIS of Environment Data Display (\mref{Viwer21})} 

\subsection{Module}
EnvDataDisplay

\subsection{Uses}
UnityEngine.UI 

\subsection{Syntax}

\subsubsection{Exported Constants}
None
\subsubsection{Exported Access Programs}
None

\subsection{Semantics}
This module is used to display the UI of the environment data. 
You can check Unity Text Documentation by clicking \tref.
\subsubsection{State Variables}
None
\subsubsection{Environment Variables}
windows: Computer screen used to display messages.

\subsubsection{Assumptions}

None

\subsubsection{Access Routine Semantics}
None

\subsubsection{Local Functions}
None
%%%%%%%%%%%%%%%%%%%%%%%%% EnvDataDisplay ends %%%%%%%%%%%%%

\newpage

%%%%%%%%%%% TreeParamDisplay starts %%%%%%%%%%%%%%%%%%
\section{MIS of Tree Parameters Display (\mref{Viwer22})} 

\subsection{Module}
TreeParamDisplay

\subsection{Uses}
UnityEngine.UI 

\subsection{Syntax}

\subsubsection{Exported Constants}
None
\subsubsection{Exported Access Programs}
None

\subsection{Semantics}
This module is used to display the UI of the tree parameters.
You can check Unity Text Documentation by clicking \tref.
\subsubsection{State Variables}
None
\subsubsection{Environment Variables}
windows: Computer screen used to display messages.
\subsubsection{Assumptions}
None
\subsubsection{Access Routine Semantics}
None
\subsubsection{Local Functions}
None
%%%%%%%%%%%%%%%%%%%%%%%%% TreeparamDisplay ends %%%%%%%%%%%%%

\newpage


%%%%%%%%%%% PauseIndicatorDisplay starts %%%%%%%%%%%%%%%%%%
\section{MIS of Pause Indicator (\mref{Viwer23})} 

\subsection{Module}
PauseIndicatorDisplay

\subsection{Uses}
UnityEngine.UI 

\subsection{Syntax}

\subsubsection{Exported Constants}
None
\subsubsection{Exported Access Programs}
None

\subsection{Semantics}
This module is used to display the status of pausing. You can
check Unity Text Documentation by clicking \tref.

\subsubsection{State Variables}
None

\subsubsection{Environment Variables}

windows: Computer screen used to display messages.

\subsubsection{Assumptions}

None

\subsubsection{Access Routine Semantics}
None

\subsubsection{Local Functions}
None
%%%%%%%%%%%%%%%%% PauseIndicatorDisplay ends %%%%%%%%%%

\newpage

\section{MIS of SeasonChangeButton (\mref{Viwer24})} 

\subsection{Module}
SeasonChangeButton

\subsection{Uses}
UnityEngine.UI , \mref{Controller27}

\subsection{Syntax}

\subsubsection{Exported Constants}
None
\subsubsection{Exported Access Programs}
None

\subsection{Semantics}
This module is used to display the seasonal change of the models of the forest

\subsubsection{State Variables}
None

\subsubsection{Environment Variables}

windows: Computer screen used to display messages.

\subsubsection{Assumptions}

None

\subsubsection{Access Routine Semantics}
None

\subsubsection{Local Functions}
None

\newpage

\section{MIS of pieChartButton (\mref{Viwer25})} 

\subsection{Module}
pieChartButton

\subsection{Uses}
UnityEngine.UI

\subsection{Syntax}

\subsubsection{Exported Constants}
None
\subsubsection{Exported Access Programs}
None

\subsection{Semantics}
This module is providing a button GUI for users to switch between a pie chart and environmental
data.

\subsubsection{State Variables}
None

\subsubsection{Environment Variables}

windows: Computer screen used to display messages.

\subsubsection{Assumptions}

None

\subsubsection{Access Routine Semantics}
None

\subsubsection{Local Functions}
None

\newpage

\section{MIS of TreeSwitchButton (\mref{Viwer26})} 

\subsection{Module}
TreeSwitchButton
\subsection{Uses}
UnityEngine.UI

\subsection{Syntax}

\subsubsection{Exported Constants}
None
\subsubsection{Exported Access Programs}
None

\subsection{Semantics}
This module provides a button GUI for users to switch between tree parameters and leaf infor
mation

\subsubsection{State Variables}
None

\subsubsection{Environment Variables}

windows: Computer screen used to display messages.

\subsubsection{Assumptions}

None

\subsubsection{Access Routine Semantics}
None

\subsubsection{Local Functions}
None

\newpage

\bibliographystyle {plainnat}
\bibliography {References}

\end{document}