\documentclass{article}

\usepackage{booktabs}
\usepackage{tabularx}
\usepackage{hyperref}
\usepackage{multirow}
\usepackage{graphicx}
\usepackage{float}
\usepackage{geometry}
\usepackage{soul}
\usepackage{ulem}
\usepackage{float}
\usepackage{multirow}


\geometry{margin = 0.75in}

\hypersetup{
    colorlinks=true,       % false: boxed links; true: colored links
    linkcolor=red,          % color of internal links (change box color with linkbordercolor)
    citecolor=green,        % color of links to bibliography
    filecolor=magenta,      % color of file links
    urlcolor=cyan           % color of external links
}
\title{Hazard Analysis\\Digital Twin Forest}

\author{Team 8, Forest Mirror
		\\ Yichen Jiang, jiany34
		\\ Bowen Zhang, zhangb82
		\\ Junhong Chen, chenj297
		\\ Jiacheng Wu, wuj187
		\\ Tingyu Shi, shit19
}

\date{}


\begin{document}


\maketitle
\thispagestyle{empty}

\pagenumbering{roman}

\begin{table}[hp]
\caption{Revision History} \label{TblRevisionHistory}
\begin{tabularx}{\textwidth}{llX}
\toprule
\textbf{Date} & \textbf{Developer(s)} & \textbf{Change}\\
\midrule
October 18 & All team members & Initial documents\\
\hline
April 2 & All team members & Final Draft \\
\bottomrule
\end{tabularx}
\end{table}




~\newpage

\tableofcontents
\listoftables
\listoffigures
\cleardoublepage

\pagenumbering{arabic}

\section{Introduction}

This document is the hazard analysis of the project Digital Twin 
Forest. Digital twin forest is a virtual representation of the 
natural world, specifically a real forest. The detailed introduction
of our project can be found \href{https://github.com/wuj187/DigitalTwinCAS/blob/main/docs/ProblemStatementAndGoals/ProblemStatement.pdf}{here}.

\section{Scope and Purpose of Hazard Analysis}

The scope of this document is to analyze and identify the hazards 
that may occur in our system  in order to minimize the cost and harm
of them when they occur.\\

\noindent Safety is always the top requirement of ours. Though we 
are developing software that may not lead to any physical harm to 
our users, we would make an effort to examine and then avoid any
other kind of possible harm. We care about not only our users but 
also our development and maintenance team, which will
provide long-term support for our users.\\

\noindent The purpose of our product is to display visualized 
forest information to users. As a result, users would be able to 
better make their strategies based on the information we provided in
the product, which means, inaccurate data might mislead the users 
when making decisions. Thus, different with many other products, 
information accuracy is another one of our top concerns. Regarding 
to the purpose of the product and the requirements of
information accuracy, we will also include the analysis of the 
possible sources of data inaccuracy in this document. \\

\noindent With the hazard analysis recorded in this document, we 
expect our product to avoid mentioned possible hazards and provide a
safer user experience for our users. 100\% safety is not achievable
in a project. However, we believe that it is still essential to 
avoid predictable unsafe situations and try our best to prepare for
unpredictable ones. In this document of hazard analysis, it
identifies unsafe behaviors while accomplishing the application, and
it makes sure such behaviors are eliminated. We would raise several
recommended actions for each hazard we mentioned, which provides 
possible solutions to avoid hazards. 

\newpage

\section{System Boundaries and Components}
\subsection{System Boundaries}
Our system includes the following contents:
\begin{itemize}
    \item Entire application
    \item Computers or laptops that execute our application
    \item JSON files to store the data
\end{itemize}
\subsection{System Components}
The application will be divided into the following 5 components, in order of importance:
\begin{enumerate}
    \item Forest Model Construction
    \item Data Visualization
    \item Data Storage
    \item Data Update from users
\end{enumerate}

\section{Critical Assumptions}
We assume that Unity game engine will always work properly.

\newpage

\section{Failure Mode and Effect Analysis}


\begin{table}[H]
\centering
\resizebox{\textwidth}{!}{
\begin{tabular}{|p{5cm}|p{3cm}|p{3cm}|p{3cm}|p{3cm}|p{1cm}|p{1cm}|}
\hline
Component& Failure Modes& Effects of Failure& Causes of Failure & 
Recommended Action & SR & Ref \\ \hline
\multirow{4}{*}{Forest Model Construction} 
& 
System lag or crash when loading tree models
& 
Users cannot use the system anymore
&
\begin{itemize}
\item Excessive accuracy of the model
\item Loading too many models at the same time
\end{itemize}
&
\begin{itemize}
\item Reduce the accuracy of the model as a trade-off
\item Use parametric modeling to solve the problem of loading 
too many models at the same time
\end{itemize}
&
SR 2.2
&
H 1-1
\\ \hline
\end{tabular}%
}
\caption{FMEA Forest Model Construction}
\end{table}

% data presentation
\begin{table}[H]
\centering
\resizebox{\textwidth}{!}{
\begin{tabular}{|p{5cm}|p{3cm}|p{3cm}|p{3cm}|p{3cm}|p{1cm}|p{1cm}|}
\hline
Component         & Failure Modes & Effects of Failure & Causes of Failure & Recommended Action & SR & Ref \\ \hline
\multirow{2}{*}{Forest Data Visualization} &        
The user interface is missing for some of the forest data       
&          
Users will not be able to collect or analyze the data if 
the interface is not presented.          
&          
The space is not large enough for the interface or the Unity UI 
the system does not work properly.         
&          
Check if the Unity UI panel and buttons are finished. Make sure the space is big enough to present all data.          
&  
SR2.4, SR2.5 
&  
H2-1    
\\\cline{2-7} 
&
UI was implemented, but the corresponding forest data are inaccurate or missing.      &
The system may crash due to calculations using non-exist or wrong data.
&
The lab and the team do not have or upload the latest forest data.
&
Download the latest forest data from ArcGIS and update the data.
Add scripts for fault tolerance when the data is missing.
&
SR2.2, SR2.5
&
H2-2  \\ \hline
\end{tabular}
}
\caption{FMEA Forest Data Visualization}
\end{table}


\begin{table}[H]
\centering

\resizebox{\textwidth}{!}{

\begin{tabular}{|p{5cm}|p{3cm}|p{3cm}|p{3cm}|p{3cm}|p{1cm}|p{1cm}|}
\hline
Component& Failure Modes& Effects of Failure& Causes of Failure & 
Recommended Action & SR & Ref \\ \hline
\multirow{4}{*}{Data Storage} 
& 
Data Storage classes were unintentionally deleted.
& 
\begin{itemize}
\item Data analysis fails
\item User interfaces fail to represent data
\end{itemize}
&
Missing data storage
&
Users can download the backup data storage  from GitHub of our
project.(As we cannot prevent the users from deleting the data.)
&
SR1.1, SR1.2 SR2.2 SR2.3
&
H 3-1 \\ \hline
\end{tabular} 

}
\caption{FMEA Data Storage}
\end{table}


\begin{table}[H]
\centering
\resizebox{\textwidth}{!}{
\begin{tabular}{|p{5cm}|p{3cm}|p{3cm}|p{3cm}|p{3cm}|p{1cm}|p{1cm}|}
\hline
Component& Failure Modes& Effects of Failure& Causes of Failure & 
Recommended Action & SR & Ref \\ \hline
\multirow{4}{*}{Data update from users} 
& 
Inappropriate user inputs
& 
\begin{itemize}
\item  Possible application crash  caused by invalid data input.
\item b. Invalid data input may  disable some of the features like
pie charts, due to the failure of necessary calculations
\end{itemize}
&
Invalid input with wrong data type
&
Implement a feature to check the data type of the input before
forwarding it to the storage.
&
SR2.2 SR2.3
&
H 4-1
\\ \hline
\end{tabular}
}
\caption{FMEA Data Update}
\end{table}

\newpage

\section{Safety and Security Requirements}
New added requirements are shown in \textcolor{blue}{blue} and will be
updated into SRS document. 
\subsection{Access Requirements}
\begin{enumerate}
\item[SR1.1] The product shall only be accessed by users who download the product from our
 GitHub website.\\
 \textcolor{blue}{
 \item[SR1.2]Forest
data shall only be modified through the interface provided by developers.\\}
\end{enumerate}
\subsection{Integrity Requirements}
\begin{enumerate}
\item[SR2.1] The system shall not propagate errors throughout the users' devices in case of failure.\\
\textcolor{blue}{
\item[SR2.2] The product shall avoid crashing when
  being used.\\
\item[SR2.3] The product shall check if user
updates(user inputs) are legal before updating them to the system.\\
\item[SR2.4] Data displayed in the application shall
 be consistent with the data stored.\\
\item[SR2.5] The system shall provide one-to-one
 mapping relationships between each data and GUI.\\}
\end{enumerate}
\subsection{Privacy Requirements}
\begin{enumerate}
    \item[SR3.1] The product shall not ask the users to provide personal information.\\
    \item[SR3.2] The product shall not send notifications to the users without permission.\\
\end{enumerate}

\subsection{Audit Requirements}
 N/A
\subsection{Immunity Requirements}
N/A
\\

\noindent 
\textcolor{red}{The corresponding fit criteria can be found 
in the SRS document}

\newpage

\section{Roadmap}
\subsection{Data storage}
Unsafe behaviours related to data storage are identified and
eliminated before hazard analysis revision 0 on October 19. This component has
the greatest importance in the failure mode and effect analysis. The team
accomplished most of the measurement during the reading week with Dr.
Gonsamo's guidance.

\subsection{Forest Model Construction}
Unsafe behaviours related to modeling will be identified and solved before the
proof of concept demonstration on November 14. The model is the basic visual
representation of the real forest. Failures may occur while the team using the
technique of parametric modeling in Unity. It is necessary to eliminate or
mitigate all of them during the early phase of the project.

\subsection{Data visualization}
Unsafe behaviours related to data visualization will be identified and solved before the POC demonstration on November 14. The team may encounter challenges and
unpredictable failures while implementing the user interface of the project. Actions shall be taken to ensure the user experience of the demo is satisfying.

\subsection{Data update}
The solution to unsafe behaviours related to data updates will
be postponed until the end of the project. According to the failure mode and effect analysis, this component is the least important. Therefore, the team shall focus on the implementation of the project first, and defer the mitigation of unsafe maintenance behaviours.
\end{document}
